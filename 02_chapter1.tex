\section{Chapter 1}

In this section we introduce the main statistical framework of this thesis. The material of this section is taken and integrated from \cite{huang2021}, except when stated otherwise.

We consider a random variable or vector $\mathbf W$ that takes value in $\mathbb V=\R^\tau$. We are interested in estimating a function $W^p[a,b]\ni\eta_0: [a,b] \rightarrow \R$ that depends on the distribution of $\mathbf W$. With $W^p[a,b]$ we denote the Sobolev space of order $p$ on $[a,b]$, that is
\begin{equation*}
    W^p[a,b] = \{h: h^{(p-1)}\text{ is absolutely continuous and }h^{(p)}\in L_2[a,b]\}.
\end{equation*}
For the ease of notation, from now on we will denote $W^p[a,b]$ as $W^p$.

We assume to have available $n$ iid copies of $\mathbf{W}$, denoted as $\mathbf{W}_1 , \ldots, \mathbf{W}_n$. The idea is to approximate $\eta_0$ with a function $\hat\eta_n\in\mathbb G\subseteq W^p[a,b]$, where $\mathbb G$ is a finite-dimensional function space that we will define later. 

We consider the utility function $\ell: W^p\times\mathbb V^n\rightarrow\R$ defined as
\begin{equation*}
    \ell(h; \mathbf{W}_1 , \ldots, \mathbf{W}_n) = \frac{1}{n}\sum_{i=1}^n l(h; \mathbf{W}_i)
\end{equation*}
where $h\in W^p$ and $l(h; \mathbf{W}_i)$ is the contribution of $\mathbf{W}_i$ to the utility.

We define the expected utility as
\begin{equation*}
    \Lambda(h) = \E\big[\ell(h; \mathbf{W}_1 , \ldots, \mathbf{W}_n)\big] = \E\big[l(h; \mathbf{W})\big].
\end{equation*}
We say that $\ell$ is a valid utility function if
\begin{enumerate}[label=\roman*)]
    \item $\displaystyle \min_{h\in W^p} \Lambda(h) = \Lambda(\eta_0)$;
    
    \item $\ell(h; \mathbf{W}_1, \ldots, \mathbf{W}_n)$ is concave in $h$, that is
    \begin{align*}
        &\ell(\alpha h_1 + (1-\alpha)h_2; \mathbf{W}_1, \ldots, \mathbf{W}_n) \\
        &\geq \alpha \ell(h_1; \mathbf{W}_1, \ldots, \mathbf{W}_n) + (1-\alpha)\ell(h_2; \mathbf{W}_1, \ldots, \mathbf{W}_n),
    \end{align*}
    for all $h_1, h_2\in\mathbb G$, for all $0 \leq \alpha \leq 1$, for all possible values of $\mathbf{W}_1, \ldots, \mathbf{W}_n$;
    
    \item $\Lambda(h)$ is strictly concave in $h$, that is
    \begin{equation*}
        \Lambda(\alpha h_1 + (1-\alpha)h_2) 
        > \alpha \Lambda(h_1) + (1-\alpha)\Lambda(h_2),
    \end{equation*}
    for all $h_1, h_2\in\mathbb G$ with $h_1 \neq h_2$, for all $0 \leq \alpha \leq 1$.
\end{enumerate}

We define the penalized utility as
\begin{equation*}
    \p\ell(h; \mathbf{W}_1 , \ldots, \mathbf{W}_n) = \ell(h; \mathbf{W}_1 , \ldots, \mathbf{W}_n) - \lambda_n J_q(h)
\end{equation*}
where $\lambda_n$ is a penalty parameter and $J_q(h)$ is a penalty term defined as
\begin{equation*}
    J_q(h) = \int_{\mathcal U} \big(h^{(q)}(x)\big)^2 \de x = \|h^{(q)}\|_2^2
\end{equation*}
where the integer $q$ is the order of the penalty and $h^{(q)}$ is the $q$-th derivative of $h$.

Similarly, we define the penalized expected utility as
\begin{equation*}
    \p\Lambda(h) = \Lambda(h) - \lambda_n J_q(h).
\end{equation*}