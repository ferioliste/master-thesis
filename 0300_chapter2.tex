\section{Methods for density estimation}\label{sec:chapter2}
In this chapter, we present and analyze several methods for probability density estimation.

We consider a continuous random variable $\mathbf X$, that takes values in $[a,b]$, with unknown density function $f_0\in W^p$. Assume we have available $n$ iid copies of $\mathbf X$, denoted as $\mathbf X_1,\dots,\mathbf X_n$. We are interested in finding an estimate $f$ of $f_0$ using this sample.

Let $\mu$ be the Lebesgue measure. We recall that, in order to be a density, $f$ needs to satisfy the following two properties:
\begin{enumerate}
    \item $f \geq 0$ $\mu$-a.e. on $[a,b]$.
    \item $\int_a^b f(x)\de x=1$
\end{enumerate}
Note that, since $f$ is continuous, the first condition is equivalent to $f(x) \geq 0$ for all $x\in [a,b]$.

In the following sections, we will derive and analyze several utility functions for probability density estimation. In order to define the utility functions, we will first define the expected utility and then we will obtain the utility by replacing expectations with averages.

In order to verify that the presented utility functions are valid we use the following sufficient condition.

\begin{proposition}[Sufficient condition for validity of utility functions]\label{prop:validity}
Let $\{\varphi_0,\ldots,\varphi_{m+k}\}$ be a basis of the spline space $\mathbb G$. For any spline $g\in\mathbb G$ there exists a unique coefficient vector $\theta\in\R^{m+k+1}$ such that
\begin{equation*}
    g(x) = \theta^\tr \varphi(x) = \sum_{i=0}^{m+k} \theta_i\varphi_i(x),
\end{equation*}
where $\varphi(x)\in\R^{m+k+1}$ is such that $\big(\varphi(x)\big)_i = \varphi_i(x)$ for $i=0,\ldots,m+k$.

We define
\begin{equation*}
    \tilde\ell(\theta;\mathbf W_1,\ldots,\mathbf W_n) = \ell\big(\theta^\tr \varphi(x);\mathbf W_1,\ldots,\mathbf W_n\big)\qquad\text{and}\qquad\tilde\Lambda(\theta) = \Lambda\big(\theta^\tr \varphi(x)\big)
\end{equation*}
for $\theta\in\R^{m+k+1}$.

Assume the following conditions hold.
\begin{enumerate}[label=\roman*)]
    \item $\displaystyle \min_{h\in W^p} \Lambda(h) = \Lambda(\eta_0)$;
    \item The Hessian $\nabla^2_{\theta} \tilde\ell(\theta;\mathbf W_1,\ldots,\mathbf W_n)$ of $\tilde\ell$ is well-defined and negative semi-definite for all $\theta\in\R^{m+k+1}$ and for all possible values of $\mathbf W_1,\ldots,\mathbf W_n$;
    \item The Hessian $\nabla^2_{\theta} \tilde\Lambda(\theta)$ of $\tilde\Lambda$ is well-defined and strictly negative definite for all $\theta\in\R^{m+k+1}$.
\end{enumerate}

Then $\ell$ is a valid utility function.
\end{proposition}

\begin{proof}\mbox{}

We verify the three properties of a valid utility function. The first property is automatically satisfied.

Let $h_1,h_2\in\mathbb G$ and $0\le\alpha\le1$. Let $\theta_1,\theta_2\in\mathbb R^{m+k+1}$ the coefficient vectors of $h_1$ and $h_2$, respectively. Since $\mathbb G$ is a linear space,
\begin{equation*}
    \alpha h_1 + (1-\alpha)h_2
= (\alpha\theta_1 + (1-\alpha)\theta_2)^\tr \varphi(x).
\end{equation*}
For the second property, we have
\begin{equation*}
    \ell(\alpha h_1 + (1-\alpha)h_2;\mathbf W_1,\ldots,\mathbf W_n)
= \tilde\ell(\alpha\theta_1 + (1-\alpha)\theta_2;
\mathbf W_1,\ldots,\mathbf W_n).
\end{equation*}
By assumption (ii), the Hessian
$\nabla^2_{\theta}\tilde\ell(\theta;\mathbf W_1,\ldots,\mathbf W_n)$
is negative semi-definite for all $\theta$ and all possible values of $\mathbf W_1,\ldots,\mathbf W_n$.
Therefore, $\tilde\ell$ is concave on $\mathbb R^{m+k+1}$, which implies
\begin{equation*}
    \tilde\ell(\alpha\theta_1 + (1-\alpha)\theta_2;\mathbf W_1,\ldots,\mathbf W_n)
    \geq
    \alpha \tilde\ell(\theta_1;\mathbf W_1,\ldots,\mathbf W_n) + (1-\alpha)\tilde\ell(\theta_2;\mathbf W_1,\ldots,\mathbf W_n)
\end{equation*}
that is the same as
\begin{equation*}
    \ell(\alpha h_1 + (1-\alpha)h_2;\mathbf W_1,\ldots,\mathbf W_n)
\geq
\alpha \ell(h_1;\mathbf W_1,\ldots,\mathbf W_n)
+ (1-\alpha)\ell(h_2;\mathbf W_1,\ldots,\mathbf W_n).
\end{equation*}

For the third property, we have
\begin{equation*}
    \Lambda(\alpha h_1 + (1-\alpha)h_2)
= \tilde\ell(\alpha\theta_1 + (1-\alpha)\theta_2).
\end{equation*}
By assumption (iii), the Hessian
$\nabla^2_{\theta}\tilde\Lambda(\theta)$
is strictly negative definite for all $\theta$.
Therefore, $\tilde\Lambda$ is strictly concave on $\mathbb R^{m+k+1}$, which implies that, for $\theta_1\neq\theta_2$,
\begin{equation*}
    \tilde\Lambda(\alpha\theta_1 + (1-\alpha)\theta_2)
    >
    \alpha \tilde\Lambda(\theta_1) + (1-\alpha)\tilde\Lambda(\theta_2)
\end{equation*}
that is the same as
\begin{equation*}
    \Lambda(\alpha h_1 + (1-\alpha)h_2)
>
\alpha \Lambda(h_1)
+ (1-\alpha)\Lambda(h_2),
\end{equation*}
for $h_1\neq h_2$.
\end{proof}

\subsection{Simple Maximum Likelihood}\label{sec:simple_MLE}

In this section, we present the classical maximum likelihood estimator of the density. The method is part of the applications of the statistical framework in \cite{huang2021}.

Let $\mathcal D$ be the space of density functions on $[a,b]$, that is
\begin{equation*}
    \mathcal D = \big\{ f: [a,b]\rightarrow \R\ \big|\ f\geq0\ \mu\text{-a.s.},\ {\textstyle \int_a^b f(x)\,\de x=1}\big\}.
\end{equation*}
The maximum likelihood method consists in maximizing $\E\big[\ln\!\big(f(\mathbf X)\big)\big]$ over all $f\in\mathcal D$, where $\mathbf X$ has distribution $f_0$. In the next proposition we show that this is indeed a consistent estimator of $f_0$.

\begin{proposition}
Fix $f_0\in\mathcal D$ and let $\mathbf X$ be a random variable with distribution $f_0$. Then
\begin{equation*}
    \argmax_{f\in\mathcal D} \E\big[\ln\!\big(f(\mathbf X)\big)\big] = f_0\ \ \ \mu\text{-a.s.}\ .
\end{equation*}
\end{proposition}

\begin{proof}\mbox{}

For $f\in\mathcal D$, we have
\begin{equation*}
    \E\big[\ln\!\big(f(\mathbf X)\big)\big] = \int_a^b \ln\!\big(f(x)\big)f_0(x)\de x = \int_a^b \ln\!\big(f_0(x)\big)f_0(x)\de x + \int_a^b \ln\!\bigg(\frac{f(x)}{f_0(x)}\bigg)f_0(x)\de x
\end{equation*}
We use the fact that for all $t > 0$, $\ln(t) \leq t - 1$. We have
\begin{gather*}
    \int_a^b \ln\!\bigg(\frac{f(x)}{f_0(x)}\bigg)f_0(x)\de x \leq \int_a^b \bigg(\frac{f(x)}{f_0(x)}-1\bigg)f_0(x)\de x \\
    = \int_a^b \big(f(x)-f_0(x)\big)\de x = 1-1 = 0,
\end{gather*}
so that
\begin{equation*}
    \E\big[\ln\!\big(f(\mathbf X)\big)\big] \leq \E\big[\ln\!\big(f_0(\mathbf X)\big)\big].
\end{equation*}
In addition, the upper bound is attained if and only if $f=f_0\ \mu$-a.s. .
\end{proof}

Since in our case $f_0\in W_p$, we would need to maximize $\E\big[\ln\!\big(f(\mathbf X)\big)\big]$ over all $f\in W_p\cap\mathcal D$. This is however unpractical in practice. In order to remove the constraints over $f$, we represent it as
\begin{equation*}
    f(\cdot) = \exp(\eta(\cdot))\Big/ \int_a^b \exp \eta(x)\de x,
\end{equation*}
with $\eta\in W_p$. This however creates an identifiability problem, since, for example, $\eta + c$ and $\eta$ would give the same density function for any constant $c$. To solve this issue, we impose
\begin{equation}\label{eq:cond_eta}
    \int_a^b\eta(x)\de x = 0.
\end{equation}

We can show that this condition is indeed enough to have a one-to-one correspondence between $\eta$ and $f$. Assume there are two functions $\eta$ and $\eta^\star$ that satisfy \autoref{eq:cond_eta} and are such that
\begin{equation*}
    f(\cdot) \overset{\mu\text{-a.e.}}= \frac{\exp(\eta(\cdot))}{\int_a^b\exp(\eta(x))\de x} \overset{\mu\text{-a.e.}}= \frac{\exp(\eta^\star(\cdot))}{\int_a^b\exp(\eta^\star(x))\de x}.
\end{equation*}
Then
\begin{equation*}
    \exp(\eta(\cdot) - \eta^\star(\cdot)) \overset{\mu\text{-a.e.}}= \frac{\exp(\eta(\cdot))}{\exp(\eta^\star(\cdot))} \overset{\mu\text{-a.e.}}= \frac{\int_a^b\exp(\eta(x))\de x}{\int_a^b\exp(\eta^\star(x))\de x} =: c
\end{equation*}
and
\begin{equation*}
    \eta(\cdot) - \eta^\star(\cdot) \overset{\mu\text{-a.e.}}= \ln(c).
\end{equation*}

Using \autoref{eq:cond_eta}, we have
\begin{equation*}
    0 = \int_a^b\eta(x)\de x - \int_a^b\eta^\star(x)\de x = \int_a^b(\eta(x)-\eta^\star(x))\de x = \int_a^b\ln(c)\de x = (b-a) \ln(c)
\end{equation*}
and so
\begin{equation*}
    0 = \ln(c) \overset{\mu\text{-a.e.}}= \eta(\cdot) - \eta^\star(\cdot).
\end{equation*}

We are now ready to define $\Lambda$ and $\ell$. We define $\Lambda$ as
\begin{equation}\label{eq:Lambda_simple_MLE}
    \Lambda(\eta) = \E\big[\ln\!\big(f(\mathbf X)\big)\big] = \E\big[\eta(\mathbf X)\big] - \ln \int_a^b \exp \eta(x)\de x,
\end{equation}
so that $\ell$ can be defined as
\begin{equation}\label{eq:ell_simple_MLE}
    \ell(\eta; \mathbf X_1 , \ldots, \mathbf X_n) = \frac{1}{n}\sum_{i=1}^n l(\eta; \mathbf X_i) = \frac{1}{n}\sum_{i=1}^n \bigg(\eta(\mathbf X_i) - \ln \int_a^b \exp \eta(x)\de x\bigg).
\end{equation}

We assume $f_0$ is bounded away from $0$ and infinity. This means that there exist $C_1^*$ and $C_2^*$ such that
\begin{equation}\label{eq:f_bounded}
    C_1^*\leq f_0\leq C_2^*\ \ \mu\text{-a.e.}\ .
\end{equation}

We will show that $\Lambda$, $\ell$ and $l$ satisfy the hypotheses of Lemmas $\ref{lem:cond_1}$, \ref{lem:cond_3}, and \ref{lem:cond_2}, so that we are able to apply Theorems \ref{thm:approx_error} and \ref{thm:est_error}. The following three propositions and their proofs are originally taken and integrated from \cite{huang2021}.

\begin{proposition}\label{prop:simple_MLE_cond_1}
    The functional $\Lambda$ as in \autoref{eq:Lambda_simple_MLE} satisfies the hypothesis of \autoref{lem:cond_1}, and so Condition 1 of \autoref{thm:approx_error} is satisfied.
\end{proposition}

\begin{proof}\mbox{}

We have that
\begin{gather*}
    \Lambda(h_1 + \alpha h_2) = \E[h_1(\mathbf X)] + \alpha\E[h_2(\mathbf X)] - \log \int_a^b \exp\big(h_1(x) + \alpha h_2(x)\big)\de x, \\
    \frac{\de}{\de\alpha}\Lambda(h_1 + \alpha h_2) = \E[h_2(\mathbf X)] - \frac{\int_a^b h_2(x)\exp\big(h_1(x) + \alpha h_2(x)\big)\de x}{\int_a^b \exp\big(h_1(x) + \alpha h_2(x)\big)\de x},\text{ and} \\
    \frac{\de^2}{\de\alpha^2}\Lambda(h_1 + \alpha h_2) \\
    = \bigg(\frac{\int_a^b h_2(x)\exp\big(h_1(x) + \alpha h_2(x)\big)\de x}{\int_a^b \exp\big(h_1(x) + \alpha h_2(x)\big)\de x}\bigg)^2 - \frac{\int_a^b h_2(x)^2\exp\big(h_1(x) + \alpha h_2(x)\big)\de x}{\int_a^b \exp\big(h_1(x) + \alpha h_2(x)\big)\de x}.
\end{gather*}

We define the random variable $\mathbf X_\alpha$ that has density
\begin{equation*}
    f_{\mathbf X_\alpha}(x) = \exp\big(h_1(x) + \alpha h_2(x)\big) \Big/ \int_a^b \exp\big(h_1(x) + \alpha h_2(x)\big)\de x.
\end{equation*}
Then,
\begin{equation}\label{eq:second_Lambda}
    \frac{\de^2}{\de\alpha^2}\Lambda(h_1 + \alpha h_2) = -\var\big(h_2(\mathbf X_\alpha) \big)
\end{equation}

Assume $0\leq\alpha\leq1$, $\|h_1\|_\infty\leq C$, and $\|h_2\|_\infty\leq B$, for $B>0$ and $C>0$. Then\linebreak $\|h_1 + \alpha h_2\|_\infty\leq B+C$. It follows that
\begin{equation*}
    f_{\mathbf X_\alpha}(x)=\frac{\exp\big(h_1(x) + \alpha h_2(x)\big)}{\int_a^b \exp\big(h_1(x) + \alpha h_2(x)\big)\de x} \leq \frac{\exp\big(B+C\big)}{\int_a^b \exp\big(-B-C\big)\de x} = \frac{\exp\big(2B+2C\big)}{b-a} = \frac{M_1}{b-a}
\end{equation*}
and
\begin{equation*}
    f_{\mathbf X_\alpha}(x)=\frac{\exp\big(h_1(x) + \alpha h_2(x)\big)}{\int_a^b \exp\big(h_1(x) + \alpha h_2(x)\big)\de x} \geq \frac{\exp\big(-B-C\big)}{\int_a^b \exp\big(B+C\big)\de x} = \frac{\exp\big(-2B-2C\big)}{b-a} = \frac{M_2}{b-a},
\end{equation*}
with
\begin{equation*}
    M_1 = \exp\big(2B+2C\big)\qquad\text{and}\qquad M_2=\exp\big(-2B-2C\big).
\end{equation*}

For fixed $c\in[a,b]$, we have
\begin{equation*}
    \frac{M_2}{b-a}\int_a^b\big(h_2(x)-c\big)^2\de x \leq \int_a^b\big(h_2(x)-c\big)^2 f_{\mathbf X_\alpha}(x)\de x \leq \frac{M_1}{b-a}\int_a^b\big(h_2(x)-c\big)^2\de x
\end{equation*}
and, taking the infima over $c$, we have
\begin{equation*}\label{eq:XalphaU}
    M_2 \var\big(h_2(\mathbf U)\big) \leq \var\big(h_2(\mathbf X_\alpha)\big) \leq M_1 \var\big(h_2(\mathbf U)\big)
\end{equation*}
where $\mathbf U$ has uniform distribution over $[a,b]$. 

Since
\begin{equation*}
    \E[h_2(\mathbf U)] = \frac{1}{b-a} \int_a^b h_2(x) \de x = 0,
\end{equation*}
we have that
\begin{equation*}
    \var\big(h_2(\mathbf U)\big) = \E\big[h_2(\mathbf U)^2\big] = \frac{1}{b-a}\int_a^b h_2(x)^2 \de x = \frac{1}{b-a}\|h_2\|_2^2.
\end{equation*}

Finally,
\begin{equation*}
     \frac{M_2}{b-a}\|h_2\|_2^2 \leq \var\big(h_2(\mathbf X_\alpha)\big) \leq \frac{M_1}{b-a}\|h_2\|_2^2
\end{equation*}
or, equivalently,
\begin{equation*}
     -\frac{M_1}{b-a}\|h_2\|_2^2 \leq \frac{\de^2}{\de\alpha^2}\Lambda(h_1 + \alpha h_2) \leq -\frac{M_2}{b-a}\|h_2\|_2^2.
\end{equation*}

Since the hypothesis is satisfied, we can use \autoref{lem:cond_1} and conclude that in this setting Condition 1 of \autoref{thm:approx_error} is satisfied.
\end{proof}

\begin{proposition}
    The function $l$ as in \autoref{eq:ell_simple_MLE} satisfies the hypothesis of \autoref{lem:cond_2}, and so Condition 1 of \autoref{thm:est_error} is satisfied.
\end{proposition}

\begin{proof}\mbox{}

We recall that
\begin{equation*}
    \dot l[\bar\eta_n; g] = \frac{\de}{\de\alpha}l(\bar\eta_n+\alpha g)\bigg|_{\alpha=0^+}.
\end{equation*}

In our case,
\begin{equation*}
    l(\eta, \mathbf X) = \eta(\mathbf X) - \ln\int_a^b\exp(\eta(x))\de x.
\end{equation*}
Then,
\begin{gather*}
    l(\bar\eta_n+\alpha g) = \bar\eta_n(\mathbf X)+\alpha g(\mathbf X) - \ln\int_a^b\exp(\bar\eta_n(x)+\alpha g(x))\de x, \\
    \frac{\de}{\de\alpha}l(\bar\eta_n+\alpha g) = g(\mathbf X) - \frac{\int_a^bg(x)\exp(\bar\eta_n(x)+\alpha g(x))\de x}{\int_a^b\exp(\bar\eta_n(x)+\alpha g(x))\de x}, \text{ and}\\
    \dot l[\bar\eta_n; g]=g(\mathbf X) - \frac{\int_a^bg(x)\exp(\bar\eta_n(x))\de x}{\int_a^b\exp(\bar\eta_n(x))\de x}.
\end{gather*}
It follows that
\begin{equation*}
    \var\big(\dot l[\bar\eta_n; g]\big) = \var\big(g(\mathbf X)\big) = \E[g(\mathbf X)^2] - \E[g(\mathbf X)]^2 \leq \E[g(\mathbf X)^2] \leq C_2^*\|g\|_2^2,
\end{equation*}
since $f_0$ is bounded away from infinity.

Since the hypothesis is satisfied, we can use \autoref{lem:cond_2} and conclude that in this setting Condition 1 of \autoref{thm:est_error} is satisfied.
\end{proof}

\begin{proposition}\label{prop:simple_MLE_cond_3}
    The function $\ell$ as in \autoref{eq:ell_simple_MLE} satisfies the hypotheses of \autoref{lem:cond_3}, and so Condition 2 of \autoref{thm:est_error} is satisfied.
\end{proposition}

\begin{proof}\mbox{}

Since Condition 1 of \autoref{thm:approx_error} is satisfied we can apply \autoref{thm:approx_error} and obtain that $\|\bar\eta_n\|_\infty$ is bounded and so Condition (i) of \autoref{lem:cond_3} is satisfied.

We have that
\begin{gather*}
    \ell(\bar\eta_n + \alpha g) = \frac{1}{n}\sum_{i=1}^n \bigg(\bar\eta_n(\mathbf X_i) + \alpha g(\mathbf X_i) - \ln\int_a^b\exp(\bar\eta_n(x) + \alpha g(x))\de x\bigg), \\
    \frac{\de}{\de\alpha}\ell(\bar\eta_n + \alpha g) = \frac{1}{n}\sum_{i=1}^n \bigg(g(\mathbf X_i) - \frac{\int_a^bg(x)\exp(\bar\eta_n(x) + \alpha g(x))\de x}{\int_a^b\exp(\bar\eta_n(x) + \alpha g(x))\de x}\bigg), \text{and}\\
    \frac{\de^2}{\de\alpha^2}\ell(\bar\eta_n + \alpha g) = \\
    -\frac{1}{n}\sum_{i=1}^n \bigg(\frac{\int_a^bg(x)^2\exp(\bar\eta_n(x) + \alpha g(x))\de x}{\int_a^b\exp(\bar\eta_n(x) + \alpha g(x))\de x} - \Big(\frac{\int_a^bg(x)\exp(\bar\eta_n(x) + \alpha g(x))\de x}{\int_a^b\exp(\bar\eta_n(x) + \alpha g(x))\de x}\Big)^2\bigg) \\
    = -\var\big(g(\bar{\mathbf X}_\alpha) \big) \overset{\star}= \frac{\de^2}{\de\alpha^2}\Lambda(\bar\eta_n + \alpha g)
\end{gather*}
where the random variable $\mathbf X_\alpha$ is as in the proof of \autoref{prop:simple_MLE_cond_1} with $h_1=\bar\eta_n$ and $h_2=g$ and $\overset{\star}=$ is justified using \autoref{eq:second_Lambda}.

It follows that $\ell(\bar\eta_n + \alpha g)$ is twice differentiable. In addition, we have
\begin{equation*}
    \frac{\de^2}{\de\alpha^2}\ell(\bar\eta_n + \alpha g) = \frac{\de^2}{\de\alpha^2}\Lambda(\bar\eta_n + \alpha g) \leq -M_2 \|g\|_2^2,\qquad\text{ for }0\leq\alpha\leq1,
\end{equation*}
since, by \autoref{prop:simple_MLE_cond_1}, $\Lambda$ satisfies the hypothesis of \autoref{lem:cond_1}.

Since the hypotheses are satisfied, we can use \autoref{lem:cond_3} and conclude that in this setting Condition 2 of \autoref{thm:est_error} is satisfied.
\end{proof}
\subsection{Shifted Maximum Likelihood}\label{sec:shifted_MLE}
In this section, we present a modified version of the classical maximum likelihood estimator of the density. This will allow us to remove the assumption that $f_0$ is bounded away from $0$, at the cost of having a (slightly) biased estimator.

For $0<\gamma<1$, we consider the random variable $\widetilde{\mathbf X}$ with density
\begin{equation*}
    \tilde f_0(\cdot) = (1-\gamma)f_0(\cdot) + \frac{\gamma}{b-a}.
\end{equation*}
Equivalently, $\widetilde{\mathbf X}$ is distributed as
\begin{equation*}
    \widetilde{\mathbf X} \sim 
    \begin{cases}
        \mathbf X, & \text{with probability } 1-\gamma,\\[0.2em]
        \mathbf U, & \text{with probability } \gamma,
    \end{cases}
\end{equation*}
where $\mathbf X$ has density $f_0$ on $[a,b]$ and $\mathbf U$ is independent of $\mathbf X$ and uniformly distributed on $[a,b]$. We will construct an estimator of $\tilde f_0$.

As for Simple Maximum Likelihood, we represent $f \in W_p\cap\mathcal D$ as
\begin{equation*}
    f(\cdot) = \exp(\eta(\cdot))\Big/ \int_a^b \exp \eta(x)\de x,
\end{equation*}
with $\eta\in W_p$ and
\begin{equation}\label{eq:cond_eta_shifted_MLE}
    \int_a^b\eta(x)\de x = 0.
\end{equation}

Then, we define $\Lambda$ as
\begin{gather}
    \Lambda(\eta) = \E\big[\ln\!\big(f(\widetilde{\mathbf X})\big)\big] = \E\big[\eta(\widetilde{\mathbf X})\big] - \ln \int_a^b \exp \eta(x)\de x \notag\\
    = (1-\gamma)\E\big[\eta(\mathbf X)\big] + \gamma\E\big[\eta(\mathbf U)\big] - \ln \int_a^b \exp \eta(x)\de x \notag \\
    = (1-\gamma)\E\big[\eta(\mathbf X)\big] + \frac{\gamma}{b-a}\int_a^b\eta(x)\de x - \ln \int_a^b \exp \eta(x)\de x, \label{eq:Lambda_shifted_MLE}
\end{gather}
so that $\ell$ can be defined as
\begin{gather}
    \ell(\eta; \mathbf X_1 , \ldots, \mathbf X_n) = \frac{1}{n}\sum_{i=1}^n l(\eta; \mathbf X_i) \notag \\
    = \frac{1}{n}\sum_{i=1}^n \bigg((1-\gamma)\eta(\mathbf X_i) + \frac{\gamma}{b-a}\int_a^b\eta(x)\de x - \ln \int_a^b \exp \eta(x)\de x\bigg). \label{eq:ell_shifted_MLE}
\end{gather}

In this case, we do not need to assume $f_0$ is bounded away from $0$. However, we still assume that $f_0$ is bounded away from infinity. This means that there exists $C_2^*$ such that
\begin{equation}\label{eq:f_bounded_shifted_MLE}
    f_0\leq C_2^*\ \ \mu\text{-a.e.}\ .
\end{equation}

First, we show that $\ell$ is a valid utility function. Then, we verify that $\Lambda$, $\ell$ and $l$ satisfy the hypotheses of Lemmas $\ref{lem:cond_1}$, \ref{lem:cond_3}, and \ref{lem:cond_2}, so that we are able to apply Theorems \ref{thm:approx_error} and \ref{thm:est_error}.

\begin{proposition}
    The function $\ell$ as in \autoref{eq:ell_shifted_MLE} is a valid utility function.
\end{proposition}

\begin{proof}\mbox{}

We verify the hypotheses of \autoref{prop:validity} so we can conclude that $\ell$ is valid.

Condition (i) is satisfied by construction.

As in the proof of \autoref{prop:MLE_valid}, we consider the restricted spline space 
\begin{equation*}
    \mathbb G^\times = \big\{\eta\in\mathbb G:{\textstyle\int_a^b\eta(x)\de x = 0}\big\}.
\end{equation*}

We define $\tilde\ell$ and $\tilde\Lambda$ as in \autoref{prop:validity}. We have
\begin{gather*}
    \tilde\ell(\theta; \mathbf X_1 , \ldots, \mathbf X_n) \\
    = \frac{1}{n}\sum_{v=1}^n \bigg((1-\gamma)\theta^\tr \tilde\varphi(\mathbf X_v) + \frac{\gamma}{b-a}\int_a^b\theta^\tr \tilde\varphi(x)\de x - \ln \int_a^b \exp\big(\theta^\tr \tilde\varphi(x)\big)\de x\bigg)
\end{gather*}
and
\begin{equation*}
    \Lambda(\theta) = (1-\gamma)\E\big[\theta^\tr \tilde\varphi(\mathbf X)\big] + \frac{\gamma}{b-a}\int_a^b\theta^\tr \tilde\varphi(x)\de x - \ln \int_a^b \exp\big(\theta^\tr \tilde\varphi(x)\big)\de x,
\end{equation*}
for $\theta\in\R^{m+k}$ and $\{\tilde\varphi_1,\ldots, \tilde\varphi_{m+k}\}$ basis of $\mathbb G^\times$.

Differentiating twice, we find
\begin{gather*}
    \big(\H(\theta)\big)_{i,j} := \big(\nabla^2_{\theta} \tilde\ell(\theta;\mathbf W_1,\ldots,\mathbf W_n)\big)_{i,j} = \big(\nabla^2_{\theta} \tilde\Lambda(\theta)\big)_{i,j} \\
    = \frac{\int_a^b \tilde\varphi_i(x)\exp\!\big(\theta^\tr \tilde\varphi(x)\big)\de x \int_a^b \tilde\varphi_j(x)\exp\!\big(\theta^\tr \tilde\varphi(x)\big)\de x}{\Big(\int_a^b\exp\!\big(\theta^\tr \tilde\varphi(x)\big)\de x\Big)^2}
    - \frac{\int_a^b \tilde\varphi_i(x)\tilde\varphi_j(x)\exp\!\big(\theta^\tr \tilde\varphi(x)\big)\de x}{\Big(\int_a^b\exp\!\big(\theta^\tr \tilde\varphi(x)\big)\de x\Big)^2}.
\end{gather*}

We notice that $\H(\theta)$ is the exactly the same as in the proof of \autoref{prop:MLE_valid}, so the rest of the proof is identical.
\end{proof}

\begin{proposition}\label{prop:shifted_MLE_cond_1}
    The functional $\Lambda$ as in \autoref{eq:Lambda_shifted_MLE} satisfies the hypothesis of \autoref{lem:cond_1}, and so condition (1) of \autoref{thm:approx_error} is satisfied.
\end{proposition}

\begin{proof}\mbox{}

In order to prove this proposition, we only need to notice that $\Lambda$ in \autoref{eq:Lambda_shifted_MLE} is exactly what $\Lambda$ in \autoref{eq:Lambda_simple_MLE} would be if we replaced $f_0$ with $\tilde f_0$. In addition, we have that $\tilde f_0$ is bounded away from $0$ and infinity. Then, we can use \autoref{prop:simple_MLE_cond_1} and conclude.
\end{proof}

\begin{proposition}
    The function $l$ as in \autoref{eq:ell_shifted_MLE} satisfies the hypothesis of \autoref{lem:cond_2}, and so condition (1) of \autoref{thm:est_error} is satisfied.
\end{proposition}

\begin{proof}\mbox{}

We recall that
\begin{equation*}
    \dot l[\bar\eta_n; g] = \frac{\de}{\de\alpha}l(\bar\eta_n+\alpha g)\bigg|_{\alpha=0^+}.
\end{equation*}

In our case,
\begin{equation*}
    l(\eta, \mathbf X) = (1-\gamma)\eta(\mathbf X) + \frac{\gamma}{b-a}\int_a^b\eta(x)\de x - \ln \int_a^b \exp \eta(x)\de x.
\end{equation*}
Then,
\begin{gather*}
    l(\bar\eta_n+\alpha g) = (1-\gamma)\big(\bar\eta_n(\mathbf X) +\alpha g(\mathbf X)\big) \\
    + \frac{\gamma}{b-a}\int_a^b(\bar\eta_n(x)+\alpha g(x))\de x - \ln\int_a^b\exp(\bar\eta_n(x)+\alpha g(x))\de x, \\
    \frac{\de}{\de\alpha}l(\bar\eta_n+\alpha g) = (1-\gamma)g(\mathbf X) + \frac{\gamma}{b-a}\int_a^b g(x)\de x - \frac{\int_a^bg(x)\exp(\bar\eta_n(x)+\alpha g(x))\de x}{\int_a^b\exp(\bar\eta_n(x)+\alpha g(x))\de x}, \text{ and}\\
    \dot l[\bar\eta_n; g]=(1-\gamma)g(\mathbf X) + \frac{\gamma}{b-a}\int_a^b g(x)\de x - \frac{\int_a^bg(x)\exp(\bar\eta_n(x))\de x}{\int_a^b\exp(\bar\eta_n(x))\de x}.
\end{gather*}
It follows that
\begin{gather*}
    \var\big(\dot l[\bar\eta_n; g]\big) = \var\big((1-\gamma)g(\mathbf X)\big) \leq \var\big(g(\mathbf X)\big) \\
    = \E[g(\mathbf X)^2] - \E[g(\mathbf X)]^2 \leq \E[g(\mathbf X)^2] \leq C_2^*\|g\|_2^2,
\end{gather*}
since $f_0$ is bounded away from infinity.

Since the hypothesis is satisfied, we can use \autoref{lem:cond_2} and conclude that in this setting condition (1) of \autoref{thm:est_error} is satisfied.
\end{proof}

\begin{proposition}
    The function $\ell$ as in \autoref{eq:ell_shifted_MLE} satisfies the hypotheses of \autoref{lem:cond_3}, and so condition (2) of \autoref{thm:est_error} is satisfied.
\end{proposition}

\begin{proof}\mbox{}

Since condition (1) of \autoref{thm:approx_error} is satisfied we can apply \autoref{thm:approx_error} and obtain that $\|\bar\eta_n\|_\infty$ is bounded and so condition (i) of \autoref{lem:cond_3} is satisfied.

Following a similar procedure as in the proof of \autoref{prop:simple_MLE_cond_3}, we can show that
\begin{equation*}
    \frac{\de^2}{\de\alpha^2}\ell(\bar\eta_n + \alpha g) = \frac{\de^2}{\de\alpha^2}\Lambda(\bar\eta_n + \alpha g).
\end{equation*}
It follows that $\ell(\bar\eta_n + \alpha g)$ is twice differentiable.

In addition, since, by \autoref{prop:simple_MLE_cond_1}, $\Lambda$ satisfies the hypothesis of \autoref{lem:cond_1}, we have
\begin{equation*}
    \frac{\de^2}{\de\alpha^2}\ell(\bar\eta_n + \alpha g) = \frac{\de^2}{\de\alpha^2}\Lambda(\bar\eta_n + \alpha g) \leq -M_2 \|g\|_2^2,\qquad\text{ for }0\leq\alpha\leq1.
\end{equation*}

Since the hypotheses are satisfied, we can use \autoref{lem:cond_3} and conclude that in this setting condition (2) of \autoref{thm:est_error} is satisfied.
\end{proof}

We conclude this section by showing how the estimator can be made asymptotically unbiased. We consider the penalized spline estimator
\begin{equation*}
    \hat\eta_n = \argmax_{h\in \mathbb G} \p\ell(h; \mathbf{W}_1 , \ldots, \mathbf{W}_n),
\end{equation*}
so that the estimated density of $\mathbf X$ will be
\begin{equation*}
    f_1(x) = \exp(\hat\eta_n(x))\Big/ \int_a^b \exp \hat\eta_n(y)\de y,\qquad\text{for }x\in[a,b].
\end{equation*}

A natural way to make $f_1$ unbiased is to define
\begin{equation*}
    f_2(x) = \frac{f_1(x)}{1-\gamma}-\frac{1}{b-a}\frac{\gamma}{1-\gamma},\qquad\text{for }x\in[a,b].
\end{equation*}
However, this does not guarantee that $f_2(x)\geq 0$ for all $x\in[a,b]$. To solve this issue, one might be tempted to define
\begin{equation*}
    f_3(x) = 0 \vee f_2(x)\Big/ \int_a^b \big(0 \vee f_2(y)\big)\de y,\qquad\text{for }x\in[a,b].
\end{equation*}
Unfortunately, this ruins completely the regularity of $f_2$ since it introduces corner points where the first derivative is discontinuous. Instead, we define
\begin{equation*}
    f_4(x) = f_2(x) + 0 \wedge\min_{y\in[a,b]} f_2(y) \Big/ \Big(1+(b-a)\big(0 \wedge\min_{y\in[a,b]} f_2(y)\big)\Big),
\end{equation*}
where the minimum of $f_2$ over $[a,b]$ always exists since $f_2$ is continuous. Intuitively, the estimator $f_4$ is $f_2$ shifted up by the bare minimum to make $f_2$ positive for all $x\in[a,b]$. In addition, by conclusion (a) of \autoref{thm:overall_error}, we have that $f_4$ is an asymptotically unbiased estimator.
\subsection{Simple Score Matching}
In this section, we present a simple score matching method for probability density estimation. The method was first introduced by Hyv\"{a}rinen in \cite{score_matching}.

We consider a continuous random variable $\mathbf X$, that takes values in $[a,b]$, with unknown density function $f_0\in W^{p+1}$ such that $f_0(a) = f_0(b) = 0$. We define the score $\psi_0$ of $f_0$ as
\begin{equation*}
    \psi_0(x) = \frac{\de}{\de x}\ln f_0(x), \qquad\text{with }x\in[a,b],
\end{equation*}
where $\psi_0\in W^p$.

\begin{proposition}
    Let $f$, $f^* \in W^{p+1}$ be density functions on $[a,b]$. We define the respective scores $\psi$ and $\psi^*$ as
    \begin{equation*}
        \psi(x) = \frac{\de}{\de x}\ln f(x) \qquad\text{and}\qquad \psi'(x) = \frac{\de}{\de x}\ln f^*(x),
    \end{equation*}
    with $x\in[a,b]$.
    
    Then, $f=f^*$ if and only if $\psi=\psi^*$.
    
    In addition,
    \begin{equation}\label{eq:psi_to_f}
        f(x) = \frac{\exp\!\big(\int_a^x \psi(t)\de t\big)}{\int_a^b\exp\!\big(\int_a^x \psi(t)\de t\big)\de x}
    \end{equation}
\end{proposition}

\begin{proof}\mbox{}

Clearly, if $f=f^*$ then $\psi=\psi^*$.

In order to prove the other direction, we construct $f$ as a function of $\psi$. We consider the primitives of $\psi$
\begin{equation*}
    \ln f(x) = \int_a^x\psi(t)\de t + c,\qquad\text{for some }c\in\R.
\end{equation*}
Then, we have
\begin{equation}\label{eq:score_matching_f}
    f(x) = \exp\!\Big(\int_a^x\psi(t)\de t + c\Big).
\end{equation}
We use the fact that $\int_a^b f(x) \de x = 1$. We have that
\begin{equation*}
    1 = e^c\int_a^b \exp\!\Big(\int_a^x\psi(t)\de t\Big) \de x
\end{equation*}
and
\begin{equation*}
    c = -\ln\!\bigg(\int_a^b \exp\!\Big(\int_a^x\psi(t)\de t\Big) \de x\bigg).
\end{equation*}

By replacing $c$ in \autoref{eq:score_matching_f}, we obtain
\begin{equation*}
    f(x) = \frac{\exp\!\big(\int_a^x \psi(t)\de t\big)}{\int_a^b\exp\!\big(\int_a^x \psi(t)\de t\big)\de x}.
\end{equation*}
\end{proof}

The idea of score matching is to estimate $\psi_0$ and then use \autoref{eq:psi_to_f} to obtain an estimate of $f_0$.

Let $\psi\in W^p$. We want to minimize the expected squared distance between $\psi(\mathbf X)$ and $\psi_0(\mathbf X)$, that is
\begin{gather*}
    J(\psi) = \frac{1}{2} \E \Big[(\psi(\mathbf X) - \psi_0(\mathbf X))^2\Big] \\
    = \frac{1}{2}\int_a^b \psi(x)^2 f_0(x)\de x - \int_a^b \psi(x)\psi_0(x)f_0(x)\de x + \frac{1}{2}\int_a^b \psi_0(x)^2f_0(x)\de x.
\end{gather*}
The third term does not depend on $\psi$ so we can remove it, as it is a constant. The second term can be rewritten as
\begin{gather*}
    \int_a^b \psi(x)\psi_0(x)f_0(x)\de x = \int_a^b \psi(x)f_0'(x)\de x = \\
    \big[\psi(x)f_0(x)\big]_a^b - \int_a^b \psi'(x)f_0(x)\de x = - \int_a^b \psi'(x)f_0(x)\de x
\end{gather*}
where the first equality uses the fact that
\begin{equation*}
    f_0(x)\psi_0(x) = f_0(x)\frac{\de}{\de x}\ln f_0(x) = f_0(x)\frac{f_0'(x)}{f_0(x)} = f_0'(x),
\end{equation*}
the second equality is by integration by parts, and the third equality uses the fact that $f_0(a) = f_0(b) = 0$. In addition, we multiply $J(\psi)$ by $-1$ in order to turn a loss function into a utility function.

Then, we define $\Lambda$ as
\begin{gather}
    \Lambda(\psi) = - \int_a^b \psi'(x)f_0(x)\de x -\frac{1}{2}\int_a^b \psi(x)^2 f_0(x)\de x \notag \\
    = \E\bigg[\!- \psi'(\mathbf X) -\frac{1}{2}\psi(\mathbf X)^2\bigg], \label{eq:Lambda_simple_SM}
\end{gather}
so that $\ell$ can be defined as
\begin{equation}\label{eq:ell_simple_SM}
    \ell(\psi; \mathbf X_1 , \ldots, \mathbf X_n) = \frac{1}{n}\sum_{i=1}^n l(\psi; \mathbf X_i)
    = \frac{1}{n}\sum_{i=1}^n \bigg(\!- \psi'(\mathbf X_i) - \frac{1}{2}\psi(\mathbf X_i)^2\bigg).
\end{equation}

The advantage of score matching is that the utility function depends only on the score $\psi$ and its derivative evaluated at the data. In particular, we never need to numerically compute any integrals in order to normalize the density.

We show that $\ell$ is a valid utility function under the assumption that $\mathbf X$ has the whole interval $[a,b]$ as support, so that every sub-interval of $[a,b]$ can be observed with positive probability.

\begin{proposition}\label{prop:simple_sm_valid}
    If $\mathrm{supp}(\mathbf X) = [a,b]$, 
    then the function $\ell$ as in \autoref{eq:ell_simple_SM} is a valid utility function.
\end{proposition}

\begin{proof}\mbox{}

We verify the hypotheses of \autoref{prop:validity} so we can conclude that $\ell$ is valid.

Condition (i) is satisfied by construction.

We define $\tilde\ell$ and $\tilde\Lambda$ as in \autoref{prop:validity}. We have
\begin{equation*}
    \tilde\ell(\theta; \mathbf X_1 , \ldots, \mathbf X_n)
    = \frac{1}{n}\sum_{v=1}^n \bigg(\!- \theta^\tr \varphi'(\mathbf X_v) - \frac{1}{2}\big(\theta^\tr \varphi(\mathbf X_v)\big)^2\bigg)
\end{equation*}
and
\begin{equation*}
    \Lambda(\theta) = \E\bigg[\!- \theta^\tr \varphi'(\mathbf X) - \frac{1}{2}\big(\theta^\tr \varphi(\mathbf X)\big)^2\bigg]
\end{equation*}
for $\theta\in\R^{m+k+1}$ and $\{\varphi_0,\ldots, \varphi_{m+k}\}$ basis of $\mathbb G$, where $\varphi'(x)\in\R^{m+k+1}$ is such that $\big(\varphi'(x)\big)_i = \varphi_i'(x)$ for $i=0,\ldots,m+k$.

In order to verify condition (ii), we differentiate $\ell$ twice, we find
\begin{equation*}
    \big(\nabla^2_{\theta} \tilde\ell(\theta;\mathbf W_1,\ldots,\mathbf W_n)\big)_{i,j} = - \frac{1}{n}\sum_{v=1}^n \varphi_i(\mathbf X_v)\varphi_j(\mathbf X_v)
\end{equation*}
so that
\begin{equation*}
    \nabla^2_{\theta} \tilde\ell(\theta;\mathbf W_1,\ldots,\mathbf W_n) = -\frac{1}{n}\sum_{v=1}^n \varphi(\mathbf X_v)\varphi(\mathbf X_v)^\tr.
\end{equation*}
Since $\varphi(\mathbf X_v)\varphi(\mathbf X_v)^\tr$ is positive semi-definite for all $v\in{1,\ldots,n}$, $\nabla^2_{\theta} \tilde\ell(\theta;\mathbf W_1,\ldots,\mathbf W_n)$ is negative semi-definite for all $\theta\in\R^{m+k+1}$.

In order to verify condition (iii), we differentiate $\Lambda$ twice, we find
\begin{equation*}
    \big(\nabla^2_{\theta} \tilde\Lambda(\theta)\big)_{i,j} = \E\Big[\varphi_i(\mathbf X)\varphi_j(\mathbf X)\Big],
\end{equation*}
so that
\begin{equation*}
    \nabla^2_{\theta} \tilde\Lambda(\theta) = \E\big[\varphi(\mathbf X)\varphi(\mathbf X)^\tr\big].
\end{equation*}

Take $u\in\R^{m+k+1}$. Then
\begin{equation*}
    u^\tr \big(\nabla^2_{\theta} \tilde\Lambda(\theta)\big)u = \E\big[\big(u^\tr\varphi(\mathbf X)\big)^2\big] \geq 0.
\end{equation*}
If $\E\big[\big(u^\tr\varphi(\mathbf X)\big)^2\big]=0$, then $u^\tr\varphi(\mathbf X) = 0$ almost surely. For $i\in\{0,\ldots,k\}$, using the hypothesis, we have
\begin{equation*}
    \Pr(\mathbf X\in[t_i,t_{i+1}],\, u^\tr\varphi(\mathbf X) = 0) = \Pr(\mathbf X\in[t_i,t_{i+1}]) > 0.
\end{equation*}
Then the set
\begin{equation*}
    A_i = \{x\in[t_i,t_{i+1}]: u^\tr\varphi(\mathbf X)=0\}
\end{equation*}
has positive Lebesgue measure. Since $u^\tr\varphi(\mathbf X)$ is a polynomial on $[t_i,t_{i+1}]$ we must have $u^\tr\varphi(x)=0$ for all $x\in[t_i,t_{i+1}]$. It follows that $u^\tr\varphi(x)=0$ for all $x\in[a,b]$ and so $u=0$.
\end{proof}


Unfortunately, to our knowledge, $\Lambda$ does not satisfy the hypotheses of the probabilistic bounds. For example, if we try to apply \autoref{lem:cond_1}, we obtain
\begin{gather*}
    \Lambda(h_1+\alpha h_2) = -\E[h_1'(\mathbf X)] -\alpha\E[h_2'(\mathbf X)] - \frac{1}{2}\E\big[\big(h_1(\mathbf X) + \alpha h_2(\mathbf X)\big)^2\big]\qquad\text{and}\\
    \frac{\de^2}{\de\alpha^2}\Lambda(h_1+\alpha h_2) = -\E\big[h_2(\mathbf X)^2\big] = -\int_a^b h_2(x)^2f_0(x)\de x.
\end{gather*}

Since $f_0(a) = f_0(b) = 0$, in general we cannot find a constant $M_2>0$ such that
\begin{equation*}
    \frac{\de^2}{\de\alpha^2}\Lambda(h_1+\alpha h_2) \leq -M_2\|h_2\|_2^2, 
    \quad 0 \leq \alpha \leq 1.
\end{equation*}

We will partially solve this issue in the next section where we introduce the Shifted Score Matching method.
\subsection{Shifted Score Matching}
In this section, we present a modified version of the simple score matching estimator of the density. This will allow us to satisfy the hypothesis of \autoref{thm:approx_error}, at the cost of having a (slightly) biased estimator.

As for the simple score matching estimator, we assume $f_0(a) = f_0(b) = 0$. In addition, we assume that $f_0$ is bounded away from infinity. This means that there exists $C_2^*$ such that
\begin{equation}
    f_0\leq C_2^*\ \ \mu\text{-a.e.}\ .
\end{equation}

Similarly to Shifted Maximum Likelihood, for $0<\gamma<1$, we consider the random variable $\widetilde{\mathbf X}$ with density
\begin{equation*}
    \tilde f_0(\cdot) = (1-\gamma)f_0(\cdot) + \frac{\gamma}{b-a}.
\end{equation*}
Equivalently, $\widetilde{\mathbf X}$ is distributed as
\begin{equation*}
    \widetilde{\mathbf X} \sim 
    \begin{cases}
        \mathbf X, & \text{with probability } 1-\gamma,\\[0.2em]
        \mathbf U, & \text{with probability } \gamma,
    \end{cases}
\end{equation*}
where $\mathbf X$ has density $f_0$ on $[a,b]$ and $\mathbf U$ is independent of $\mathbf X$ and uniformly distributed on $[a,b]$. We will construct an estimator of $\tilde f_0$.

We define the score $\widetilde\psi_0$ of $\tilde f_0$ as
\begin{equation*}
    \widetilde\psi_0(x) = \frac{\de}{\de x}\ln \tilde f_0(x) = \frac{\de}{\de x}\ln\Big((1-\gamma)f_0(\cdot) + \frac{\gamma}{b-a}\Big), \qquad\text{with }x\in[a,b],
\end{equation*}
where $\widetilde\psi_0\in W^p$.

As in the simple score matching method, we want to estimate $\widetilde\psi_0$ and then use \autoref{eq:psi_to_f} to obtain an estimate of $\tilde f_0$.

Let $\psi\in W^p$. We want to minimize the expected squared distance between $\psi(\widetilde{\mathbf X})$ and $\psi_0(\widetilde{\mathbf X})$, that is
\begin{gather*}
    J(\psi) = \frac{1}{2} \E \Big[(\psi(\widetilde{\mathbf X}) - \widetilde\psi_0(\widetilde{\mathbf X}))^2\Big] \\
    = \frac{1}{2}\int_a^b \psi(x)^2 \tilde f_0(x)\de x - \int_a^b \psi(x)\widetilde\psi_0(x)\tilde f_0(x)\de x + \frac{1}{2}\int_a^b \widetilde\psi_0(x)^2\tilde f_0(x)\de x.
\end{gather*}
As in the simple score matching method, we can remove the third term as it does not depend on $\psi$. The second term can be rewritten as
\begin{gather*}
    \int_a^b \psi(x)\widetilde\psi_0(x)\tilde f_0(x)\de x = \int_a^b \psi(x)\tilde f_0'(x)\de x = (1-\gamma)\int_a^b \psi(x)f_0'(x)\de x = \\
    (1-\gamma)\big[\psi(x)f_0(x)\big]_a^b - (1-\gamma)\int_a^b \psi'(x)f_0(x)\de x = - (1-\gamma)\int_a^b \psi'(x)f_0(x)\de x
\end{gather*}
where the first equality uses the fact that
\begin{equation*}
    \tilde f_0(x)\widetilde\psi_0(x) = \tilde f_0(x)\frac{\de}{\de x}\ln \tilde f_0(x) = \tilde f_0(x)\frac{\tilde f_0'(x)}{\tilde f_0(x)} = \tilde f_0'(x),
\end{equation*}
the second equality is by integration by parts, and the third equality uses the fact that $f_0(a) = f_0(b) = 0$. The first term can be rewritten as
\begin{gather*}
    \frac{1}{2}\int_a^b \psi(x)^2 \tilde f_0(x)\de x = \frac{1}{2}\int_a^b \psi(x)^2 \bigg((1-\gamma)f_0(x) + \frac{\gamma}{b-a}\bigg)\de x \\
    = \frac{1}{2}(1-\gamma)\int_a^b \psi(x)^2 f_0(x)\de x + \frac{1}{2} \frac{\gamma}{b-a}\int_a^b \psi(x)^2 \de x.
\end{gather*}

As before, we multiply $J(\psi)$ by $-1$ in order to turn a loss function into a utility function.

Then, we define $\Lambda$ as
\begin{gather}
    \Lambda(\psi) = - (1-\gamma)\int_a^b \psi'(x)f_0(x)\de x -\frac{1}{2}(1-\gamma)\int_a^b \psi(x)^2 f_0(x)\de x - \frac{1}{2}\frac{\gamma}{b-a}\int_a^b \psi(x)^2 \de x \notag \\
    = \E\bigg[\!- (1-\gamma)\psi'(\mathbf X) -\frac{1}{2}(1-\gamma)\psi(\mathbf X)^2\bigg] - \frac{1}{2}\frac{\gamma}{b-a}\int_a^b \psi(x)^2 \de x, \label{eq:Lambda_shifted_SM}
\end{gather}
so that $\ell$ can be defined as
\begin{gather}
    \ell(\psi; \mathbf X_1 , \ldots, \mathbf X_n) = \frac{1}{n}\sum_{i=1}^n l(\psi; \mathbf X_i) \notag \\
    = \frac{1}{n}\sum_{i=1}^n \bigg(\!- (1-\gamma)\psi'(\mathbf X_i) - \frac{1}{2}(1-\gamma)\psi(\mathbf X_i)^2 - \frac{1}{2}\frac{\gamma}{b-a}\int_a^b \psi(x)^2 \de x\bigg). \label{eq:ell_shifted_SM}
\end{gather}

We notice that, in practice, the integral $\int_a^b \psi(x)^2\de x$ does not need to be approximated numerically. Indeed, for $\psi\in\mathbb G$, $\psi^2$ is a polynomial on each interval $[t_i,t_{i+1}]$. Therefore, the integral can be computed exactly as
\begin{equation*}
    \int_a^b \psi(x)^2\de x = \sum_{i=1}^k \int_{t_i}^{t_{i+1}} \psi(x)^2\de x.
\end{equation*}

First we show that $\ell$ is a valid utility function. Then, we verify that $\Lambda$ satisfies the hypothesis of \autoref{lem:cond_1}, so that we are able to apply \autoref{thm:approx_error}.

\begin{proposition}
    The function $\ell$ as in \autoref{eq:ell_shifted_SM} is a valid utility function.
\end{proposition}

\begin{proof}\mbox{}

We verify the hypotheses of \autoref{prop:validity} so we can conclude that $\ell$ is valid.

Condition (i) is satisfied by construction.

We define $\tilde\ell$ and $\tilde\Lambda$ as in \autoref{prop:validity}. We have
\begin{gather*}
    \tilde\ell(\theta; \mathbf X_1 , \ldots, \mathbf X_n) \\
    = \frac{1}{n}\sum_{v=1}^n \bigg(\!- (1-\gamma)\theta^\tr \varphi'(\mathbf X_v) - \frac{1}{2}(1-\gamma)\big(\theta^\tr \varphi(\mathbf X_v)\big)^2 - \frac{1}{2}\frac{\gamma}{b-a}\int_a^b \psi(x)^2 \de x\bigg)
\end{gather*}
and
\begin{equation*}
    \Lambda(\theta) = \E\bigg[\!- (1-\gamma)\theta^\tr \varphi'(\mathbf X) - \frac{1}{2}(1-\gamma)\big(\theta^\tr \varphi(\mathbf X)\big)^2\bigg] - \frac{1}{2}\frac{\gamma}{b-a}\int_a^b \psi(x)^2 \de x,
\end{equation*}
for $\theta\in\R^{m+k+1}$ and $\{\varphi_0,\ldots, \varphi_{m+k}\}$ basis of $\mathbb G$.

In order to verify condition (ii), we differentiate $\ell$ twice, we find
\begin{equation*}
    \big(\nabla^2_{\theta} \tilde\ell(\theta;\mathbf W_1,\ldots,\mathbf W_n)\big)_{i,j} = -\frac{1}{n}\sum_{v=1}^n (1-\gamma)\varphi_i(\mathbf X_v)\varphi_j(\mathbf X_v) - \frac{\gamma}{b-a}\int_a^b \varphi_i(x)\varphi_j(x)\de x
\end{equation*}
so that
\begin{equation*}
    \nabla^2_{\theta} \tilde\ell(\theta;\mathbf W_1,\ldots,\mathbf W_n) = -\frac{1}{n}\sum_{v=1}^n (1-\gamma)\varphi(\mathbf X_v)\varphi(\mathbf X_v)^\tr - \frac{\gamma}{b-a}G_0,
\end{equation*}
with $(G_0)_{i,j} = \int_a^b \varphi_i(x)\varphi_j(x)\de x$.

Take $u\in\R^{m+k+1}$. Then
\begin{gather*}
    u^\tr G_0 u = \sum_{i=0}^{m+k}\sum_{j=0}^{m+k}u_i u_j (G_0)_{i,j} = \sum_{i=0}^{m+k}\sum_{j=0}^{m+k}u_i u_j \int_a^b \varphi_i(x)\varphi_j(x)\de x \\
    = \int_a^b \bigg(\sum_{i=0}^{m+k} \theta_i\varphi_i(x)\bigg)^2 \de x \geq 0
\end{gather*}
where the equality holds if and only if $u=0$. Then, $G_0$ is strictly positive definite.

Since $\varphi(\mathbf X_v)\varphi(\mathbf X_v)^\tr$ is positive semi-definite for all $v\in{1,\ldots,n}$ and $G_0$ is strictly positive definite, $\nabla^2_{\theta} \tilde\ell(\theta;\mathbf W_1,\ldots,\mathbf W_n)$ is strictly negative definite for all $\theta\in\R^{m+k+1}$.

In order to verify condition (iii), we differentiate $\Lambda$ twice, we find
\begin{equation*}
    \big(\nabla^2_{\theta} \tilde\Lambda(\theta)\big)_{i,j} = (1-\gamma)\E\Big[\varphi_i(\mathbf X)\varphi_j(\mathbf X)\Big] - \frac{\gamma}{b-a}\int_a^b \varphi_i(x)\varphi_j(x)\de x,
\end{equation*}
so that
\begin{equation*}
    \nabla^2_{\theta} \tilde\Lambda(\theta) = (1-\gamma)\E\big[\varphi(\mathbf X)\varphi(\mathbf X)^\tr\big] - \frac{\gamma}{b-a}G_0.
\end{equation*}

Since $\E\big[\varphi(\mathbf X)\varphi(\mathbf X)^\tr\big]$ is positive semi-definite and $G_0$ is strictly positive definite, $\nabla^2_{\theta} \tilde\Lambda(\theta)$ is strictly negative definite for all $\theta\in\R^{m+k+1}$.
\end{proof}

\begin{proposition}
The functional $\Lambda$ as in \autoref{eq:Lambda_shifted_SM} satisfies the hypothesis of \autoref{lem:cond_1}, and so condition (1) of \autoref{thm:approx_error} is satisfied.
\end{proposition}

\begin{proof}\mbox{}

We have that
\begin{gather*}
    \Lambda(h_1 + \alpha h_2) = \E\bigg[\!- (1-\gamma)\big(h_1'(\mathbf X) + \alpha h_2'(\mathbf X)\big) -\frac{1}{2}(1-\gamma)\big(h_1(\mathbf X)+ \alpha h_2(\mathbf X)\big)^2\bigg] \\
    - \frac{1}{2}\frac{\gamma}{b-a}\int_a^b \big(h_1(x)+ \alpha h_2(x)\big)^2 \de x\qquad\text{and} \\
    \frac{\de^2}{\de\alpha^2}\Lambda(h_1 + \alpha h_2)
    = \E\big[\!-(1-\gamma)h_2(\mathbf X)^2\big]
    - \frac{\gamma}{b-a}\int_a^b h_2(x)^2 \de x.
\end{gather*}

Since $f_0$ is bounded away from infinity,
\begin{equation*}
    0\leq\E\big[h_2(\mathbf X)^2\big]\leq C_2^*\|h_2\|_2^2.
\end{equation*}
Then,
\begin{equation*}
    -\Big((1-\gamma)C_2^* + \frac{\gamma}{b-a}\Big)\|h_2\|_2^2 \leq \Lambda(h_1 + \alpha h_2) \leq -\frac{\gamma}{b-a}\|h_2\|_2^2.
\end{equation*}

Since the hypotheses are satisfied, we can use \autoref{lem:cond_1} and conclude that in this setting condition (2) of \autoref{thm:approx_error} is satisfied.
\end{proof}

Unfortunately, to our knowledge, $l$ still does not satisfy the hypothesis of \autoref{thm:est_error}. We report here our attempt to prove the hypothesis of \autoref{lem:cond_2_bis} and explain why it fails.

We recall that
\begin{equation*}
    \dot l[\bar\eta_n; g] = \frac{\de}{\de\alpha}l(\bar\eta_n+\alpha g)\bigg|_{\alpha=0^+}.
\end{equation*}

For $h\in\mathbb G$ with $\|h\|_2\leq 1$, we have
\begin{gather*}
    l(\bar\eta_n+\alpha g) = - (1-\gamma)\big(\bar\eta_n'(\mathbf X)+\alpha g'(\mathbf X)\big) - \frac{1}{2}(1-\gamma)\big(\bar\eta_n(\mathbf X)+\alpha g(\mathbf X)\big)^2 \\
    - \frac{1}{2}\frac{\gamma}{b-a}\int_a^b \big(\bar\eta_n(x)+\alpha g(x)\big)^2 \de x,\\
    \frac{\de}{\de\alpha}l(\bar\eta_n+\alpha g) = - (1-\gamma)g'(\mathbf X) - \frac{1}{2}(1-\gamma)g(\mathbf X)\big(\bar\eta_n(\mathbf X)+\alpha g(\mathbf X)\big) \\
    - \frac{1}{2}\frac{\gamma}{b-a}\int_a^b g(x)\big(\bar\eta_n(x)+\alpha g(x)\big) \de x, \text{ and}\\
    \dot l[\bar\eta_n; g] = - (1-\gamma)g'(\mathbf X) - \frac{1}{2}(1-\gamma)g(\mathbf X)\bar\eta_n(\mathbf X) - \frac{1}{2}\frac{\gamma}{b-a}\int_a^b g(x)\bar\eta_n(x)\de x.
\end{gather*}
It follows that
\begin{gather*}
    \var\big(\dot l[\bar\eta_n; g]\big) \leq \E\Big[\Big(- (1-\gamma)g'(\mathbf X) - \frac{1}{2}(1-\gamma)g(\mathbf X)\bar\eta_n(\mathbf X)\Big)^2\Big] \\
    \leq \E\Big[\Big(g'(\mathbf X) + \frac{1}{2}g(\mathbf X)\bar\eta_n(\mathbf X)\Big)^2\Big] \leq 2\E\big[g'(\mathbf X)^2\big] + \E\big[g(\mathbf X)^2\bar\eta_n(\mathbf X)^2\big] = 2\,\mathrm{I} + \mathrm{II}.
\end{gather*}
Using \autoref{thm:approx_error}, we know that $\bar\eta_n$ is bounded. Then there exists $C$ such that $\bar\eta_n\leq C$ $\mu$-a.s. for all $n$. Then, we have that
\begin{equation*}
    \mathrm{II} \leq C^2\E\big[g(\mathbf X)^2\big] \leq C^2C_2^*\|g\|_2^2
\end{equation*}
since $f_0$ is bounded away from infinity.

To conclude, we would only need to prove that
\begin{equation*}
    \|g'\|_2^2 \leq M\|g\|_2^2
\end{equation*}
for some constant $M>0$ that depends only on $[a,b]$ and $m$. Unfortunately, this is not possible. We construct a counterexample.

For simplicity, take $[a,b]=[0,1]$ and fix the spline degree $m = 2$. We consider the following sequence of knots vectors. For each $n\in\mathbb N$, we take the knots partition $t_{(n)} \in\R^{n+2}$ with
\begin{equation*}
    0 = (t_{(n)})_0 < (t_{(n)})_1 < \dots < (t_{(n)})_{n+1} = 1, \qquad (t_{(n)})_i := \frac{i}{n+1},\text{ for }i=0,\ldots,n+1.
\end{equation*}
Let $(\mathbb G_n)_{n\in\N}$ the sequence of spline spaces. We consider the spline $g_n\in \mathbb G_n$ defined as
\begin{equation*}
    g_n(x) = \begin{cases}
    \sqrt{5(n+1)^5}\Big(x-\frac{1}{n+1}\Big)^2\qquad\ \, \text{if }0\leq x < \frac{1}{n+1} \\
    0\qquad\qquad\qquad\qquad\qquad\qquad\ \text{if }\frac{1}{n+1}\leq x \leq 1
    \end{cases}
\end{equation*}
so that $\|g_n\|_2 = 1$ for all $n\in\N$. The derivative of $g_n$ then is
\begin{equation*}
    g_n'(x) = \begin{cases}
    2\sqrt{5(n+1)^5}\Big(x-\frac{1}{n+1}\Big)\qquad\ \, \text{if }0\leq x < \frac{1}{n+1} \\
    0\qquad\qquad\qquad\qquad\qquad\qquad\ \text{if }\frac{1}{n+1}\leq x \leq 1
    \end{cases},
\end{equation*}
so that we have $\|g'\|_2\rightarrow\infty$ as $n\rightarrow\infty$. Then, there cannot exist $M>0$ such that $\|g_n'\|_2^2 \leq M\|g_n\|_2^2 = M$ for all $n\in\N$.

However, thanks to \autoref{thm:approx_error}, we know that $f_0$ can be approximated arbitrarily well in spline spaces as the number of knots increases. Moreover, for any $\psi\in L_2[a,b]$, the law of large numbers gives that
\begin{equation*}
    \ell(\psi; \mathbf X_1 , \ldots, \mathbf X_n) \xrightarrow[n\to\infty]{\text{a.s.}} \Lambda(\psi)
\end{equation*}
Therefore, although we have not been able to verify the validity of \autoref{thm:est_error}, we expect the estimator to still perform well in practice. In general, however, we don't have guarantees on how quickly $\ell(\psi)$ converges to $\Lambda(\psi)$, as finer spline spaces may lead the variance of $\ell(\psi)$ to increase.

Finally, the estimator can be made asymptotically unbiased by following the exact same process as for the Shifted Maximum Likelihood method.
\subsection{Generalized Score matching}
In this section, we present a generalized version of the simple score matching estimator of the density. This will allow us to remove the assumption that $f_0(a) = f_0(b) = 0$. The method was first introduced by Yu et al. in \cite{gen_sm}.

We consider a continuous random variable $\mathbf X$, that takes values in $[a,b]$, with unknown density function $f_0\in W^{p+1}$. As before, we define the score $\psi_0$ of $f_0$ as
\begin{equation*}
    \psi_0(x) = \frac{\de}{\de x}\ln f_0(x), \qquad\text{with }x\in[a,b],
\end{equation*}
where $\psi_0\in W^p$.

We define the functions
\begin{equation*}
    \phi(x) = \min\!\big\{x-a;\ b-x;\ 1\big\}\qquad\text{and}\qquad \kappa(x) = x^\alpha
\end{equation*}
with $\alpha>1$.

As in the simple score matching method, we want to estimate $\psi_0$ and then use \autoref{eq:psi_to_f} to obtain an estimate of $f_0$.

Let $\psi\in W^p$. We want to minimize
\begin{gather*}
    J(\psi) = \frac{1}{2} \E \Big[\kappa\big(\phi(\mathbf X)\big)(\psi(\mathbf X) - \psi_0(\mathbf X))^2\Big] \\
    = \frac{1}{2}\int_a^b \kappa\big(\phi(x)\big)\psi(x)^2 f_0(x)\de x - \int_a^b \kappa\big(\phi(x)\big)\psi(x)\psi_0(x)f_0(x)\de x \\
    + \frac{1}{2}\int_a^b \kappa\big(\phi(x)\big)\psi_0(x)^2f_0(x)\de x.
\end{gather*}

The third term does not depend on $\psi$ so we can remove it, as it is a constant. The second term can be rewritten as
\begin{gather*}
    \int_a^b \kappa\big(\phi(x)\big)\psi(x)\psi_0(x)f_0(x)\de x = \int_a^b \kappa\big(\phi(x)\big)\psi(x)f_0'(x)\de x = \\
    \big[\kappa\big(\phi(x)\big)\psi(x)f_0(x)\big]_a^b - \int_a^b \kappa\big(\phi(x)\big)\psi'(x)f_0(x)\de x - \int_a^b \phi'(x)\kappa'\big(\phi(x)\big)\psi(x)f_0(x)\de x \\
    = - \int_a^b \kappa\big(\phi(x)\big)\psi'(x)f_0(x)\de x - \int_a^b \phi'(x)\kappa'\big(\phi(x)\big)\psi(x)f_0(x)\de x
\end{gather*}
where the first equality uses the fact that
\begin{equation*}
    f_0(x)\psi_0(x) = f_0(x)\frac{\de}{\de x}\ln f_0(x) = f_0(x)\frac{f_0'(x)}{f_0(x)} = f_0'(x),
\end{equation*}
the second equality is by integration by parts, and the third equality uses the fact that $\kappa\big(\phi(a)\big) = \kappa\big(\phi(b)\big) = 0$. In addition, we multiply $J(\psi)$ by $-1$ in order to turn a loss function into a utility function.

Then, we define $\Lambda$ as
\begin{gather}
    \Lambda(\psi) = - \int_a^b \kappa\big(\phi(x)\big)\psi'(x)f_0(x)\de x - \int_a^b \phi'(x)\kappa'\big(\phi(x)\big)\psi(x)f_0(x)\de x \notag \\
    -\frac{1}{2}\int_a^b \kappa\big(\phi(x)\big)\psi(x)^2 f_0(x)\de x \notag \\
    = \E\bigg[\!- \kappa\big(\phi(\mathbf X)\big)\psi'(\mathbf X) - \phi'(\mathbf X)\kappa'\big(\phi(\mathbf X)\big)\psi(\mathbf X) - \frac{1}{2}\kappa\big(\phi(\mathbf X)\big)\psi(\mathbf X)^2\bigg], \label{eq:Lambda_gen_SM}
\end{gather}
so that $\ell$ can be defined as
\begin{gather}
    \ell(\psi; \mathbf X_1 , \ldots, \mathbf X_n) = \frac{1}{n}\sum_{i=1}^n l(\psi; \mathbf X_i)
    \notag \\
    = \frac{1}{n}\sum_{i=1}^n \bigg(\!- \kappa\big(\phi(\mathbf X_i)\big)\psi'(\mathbf X_i) - \phi'(\mathbf X_i)\kappa'\big(\phi(\mathbf X_i)\big)\psi(\mathbf X_i) - \frac{1}{2}\kappa\big(\phi(\mathbf X_i)\big)\psi(\mathbf X_i)^2\bigg).\label{eq:ell_gen_SM}
\end{gather}

Unfortunately, similarly to the simple score matching method, $\Lambda$ does not satisfy the hypotheses of the probabilistic bounds. For example, if we try again to apply \autoref{lem:cond_1}, we obtain
\begin{gather*}
    \frac{\de^2}{\de\alpha^2}\Lambda(h_1+\alpha h_2) = -\E\big[\kappa\big(\phi(\mathbf X)\big)h_2(\mathbf X)^2\big] = -\int_a^b \kappa\big(\phi(x)\big)h_2(x)^2f_0(x)\de x.
\end{gather*}

Since $\kappa\big(\phi(a)\big) = \kappa\big(\phi(b)\big) = 0$, in general we cannot find a constant $M_2>0$ such that
\begin{equation*}
    \frac{\de^2}{\de\alpha^2}\Lambda(h_1+\alpha h_2) \leq -M_2\|h_2\|_2^2, 
    \quad 0 \leq \alpha \leq 1.
\end{equation*}