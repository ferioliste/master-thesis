\subsection{Simple Maximum Likelihood}\label{sec:simple_MLE}

In this section, we present the classical maximum likelihood estimator of the density. The method is part of the applications of the statistical framework in \cite{huang2021}.

Let $\mathcal D$ be the space of density functions on $[a,b]$, that is
\begin{equation*}
    \mathcal D = \big\{ f: [a,b]\rightarrow \R\ \big|\ f\geq0\ \mu\text{-a.s.},\ {\textstyle \int_a^b f(x)\,\de x=1}\big\}.
\end{equation*}
The maximum likelihood method consists in maximizing $\E\big[\ln\!\big(f(\mathbf X)\big)\big]$ over all $f\in\mathcal D$, where $\mathbf X$ has distribution $f_0$. In the next proposition we show that this is indeed a consistent estimator of $f_0$.

\begin{proposition}
Fix $f_0\in\mathcal D$ and let $\mathbf X$ be a random variable with distribution $f_0$. Then
\begin{equation*}
    \argmax_{f\in\mathcal D} \E\big[\ln\!\big(f(\mathbf X)\big)\big] = f_0\ \ \ \mu\text{-a.s.}\ .
\end{equation*}
\end{proposition}

\begin{proof}\mbox{}

For $f\in\mathcal D$, we have
\begin{equation*}
    \E\big[\ln\!\big(f(\mathbf X)\big)\big] = \int_a^b \ln\!\big(f(x)\big)f_0(x)\de x = \int_a^b \ln\!\big(f_0(x)\big)f_0(x)\de x + \int_a^b \ln\!\bigg(\frac{f(x)}{f_0(x)}\bigg)f_0(x)\de x
\end{equation*}
We use the fact that for all $t > 0$, $\ln(t) \leq t - 1$. We have
\begin{gather*}
    \int_a^b \ln\!\bigg(\frac{f(x)}{f_0(x)}\bigg)f_0(x)\de x \leq \int_a^b \bigg(\frac{f(x)}{f_0(x)}-1\bigg)f_0(x)\de x \\
    = \int_a^b \big(f(x)-f_0(x)\big)\de x = 1-1 = 0,
\end{gather*}
so that
\begin{equation*}
    \E\big[\ln\!\big(f(\mathbf X)\big)\big] \leq \E\big[\ln\!\big(f_0(\mathbf X)\big)\big].
\end{equation*}
In addition, the upper bound is attained if and only if $f=f_0\ \mu$-a.s. .
\end{proof}

Since in our case $f_0\in W_p$, we would need to maximize $\E\big[\ln\!\big(f(\mathbf X)\big)\big]$ over all $f\in W_p\cap\mathcal D$. This is however unpractical in practice. In order to remove the constraints over $f$, we represent it as
\begin{equation*}
    f(\cdot) = \exp(\eta(\cdot))\Big/ \int_a^b \exp \eta(x)\de x,
\end{equation*}
with $\eta\in W_p$. This however creates an identifiability problem, since, for example, $\eta + c$ and $\eta$ would give the same density function for any constant $c$. To solve this issue, we impose
\begin{equation}\label{eq:cond_eta}
    \int_a^b\eta(x)\de x = 0.
\end{equation}

We can show that this condition is indeed enough to have a one-to-one correspondence between $\eta$ and $f$. Assume there are two functions $\eta$ and $\eta^\star$ that satisfy \autoref{eq:cond_eta} and are such that
\begin{equation*}
    f(\cdot) \overset{\mu\text{-a.e.}}= \frac{\exp(\eta(\cdot))}{\int_a^b\exp(\eta(x))\de x} \overset{\mu\text{-a.e.}}= \frac{\exp(\eta^\star(\cdot))}{\int_a^b\exp(\eta^\star(x))\de x}.
\end{equation*}
Then
\begin{equation*}
    \exp(\eta(\cdot) - \eta^\star(\cdot)) \overset{\mu\text{-a.e.}}= \frac{\exp(\eta(\cdot))}{\exp(\eta^\star(\cdot))} \overset{\mu\text{-a.e.}}= \frac{\int_a^b\exp(\eta(x))\de x}{\int_a^b\exp(\eta^\star(x))\de x} =: c
\end{equation*}
and
\begin{equation*}
    \eta(\cdot) - \eta^\star(\cdot) \overset{\mu\text{-a.e.}}= \ln(c).
\end{equation*}

Using \autoref{eq:cond_eta}, we have
\begin{equation*}
    0 = \int_a^b\eta(x)\de x - \int_a^b\eta^\star(x)\de x = \int_a^b(\eta(x)-\eta^\star(x))\de x = \int_a^b\ln(c)\de x = (b-a) \ln(c)
\end{equation*}
and so
\begin{equation*}
    0 = \ln(c) \overset{\mu\text{-a.e.}}= \eta(\cdot) - \eta^\star(\cdot).
\end{equation*}

We are now ready to define $\Lambda$ and $\ell$. We define $\Lambda$ as
\begin{equation}\label{eq:Lambda_simple_MLE}
    \Lambda(\eta) = \E\big[\ln\!\big(f(\mathbf X)\big)\big] = \E\big[\eta(\mathbf X)\big] - \ln \int_a^b \exp \eta(x)\de x,
\end{equation}
so that $\ell$ can be defined as
\begin{equation}\label{eq:ell_simple_MLE}
    \ell(\eta; \mathbf X_1 , \ldots, \mathbf X_n) = \frac{1}{n}\sum_{i=1}^n l(\eta; \mathbf X_i) = \frac{1}{n}\sum_{i=1}^n \bigg(\eta(\mathbf X_i) - \ln \int_a^b \exp \eta(x)\de x\bigg).
\end{equation}

We assume $f_0$ is bounded away from $0$ and infinity. This means that there exist $C_1^*$ and $C_2^*$ such that
\begin{equation}\label{eq:f_bounded}
    C_1^*\leq f_0\leq C_2^*\ \ \mu\text{-a.e.}\ .
\end{equation}

First, we show that $\ell$ is a valid utility function. This argument is not included in \cite{huang2021} and we provide it here. Then, we verify that $\Lambda$, $\ell$, and $l$ satisfy the hypotheses of Lemmas \ref{lem:cond_1}, \ref{lem:cond_3}, and \ref{lem:cond_2}, so that we are able to apply Theorems \ref{thm:approx_error} and \ref{thm:est_error}. The three respective propositions and proofs are originally taken and integrated from \cite{huang2021}.

\begin{proposition}\label{prop:MLE_valid}
    The function $\ell$ as in \autoref{eq:ell_simple_MLE} is a valid utility function.
\end{proposition}

\begin{proof}\mbox{}

We verify the hypotheses of \autoref{prop:validity} so we can conclude that $\ell$ is valid.

Condition (i) is satisfied by construction.

We consider a basis of $\mathbb G$, $\{\varphi_0,\ldots,\varphi_{m+k}\}$, such that $\varphi_0\equiv1$. For any spline $g\in\mathbb G$ there exists a unique coefficient vector $\theta\in\R^{m+k+1}$ such that
\begin{equation*}
    g(x) = \theta^\tr \varphi(x) = \sum_{i=0}^{m+k} \theta_i\varphi_i(x).
\end{equation*}
Imposing \autoref{eq:cond_eta}, we have that
\begin{equation*}
    \theta_0 = -\frac{1}{b-a}\int_a^b \sum_{i=1}^{m+k}\theta_i\varphi_i(x)\de x,
\end{equation*}
so that the restricted spline space
\begin{equation*}
    \mathbb G^\times = \big\{\eta\in\mathbb G:{\textstyle\int_a^b\eta(x)\de x = 0}\big\}
\end{equation*}
has dimension $m+k$. In addition, we notice that, except for $0$, constants functions do not belong to $\mathbb G^\times$.

We consider a basis of $\mathbb G^\times$, $\{\tilde\varphi_1,\ldots,\tilde\varphi_{m+k}\}$, and we define $\tilde\ell$ and $\tilde\Lambda$ as in \autoref{prop:validity}. We have
\begin{equation*}
    \tilde\ell(\theta; \mathbf X_1 , \ldots, \mathbf X_n) = \frac{1}{n}\sum_{v=1}^n \bigg(\theta^\tr \tilde\varphi(\mathbf X_v) - \ln \int_a^b \exp\big(\theta^\tr \tilde\varphi(x)\big)\de x\bigg)
\end{equation*}
and
\begin{equation*}
    \Lambda(\theta) = \E\big[\theta^\tr \tilde\varphi(\mathbf X)\big] - \ln \int_a^b \exp\big(\theta^\tr \tilde\varphi(x)\big)\de x,
\end{equation*}
for $\theta\in\R^{m+k}$.

Differentiating twice, we find
\begin{gather*}
    \big(\H(\theta)\big)_{i,j} := \big(\nabla^2_{\theta} \tilde\ell(\theta;\mathbf W_1,\ldots,\mathbf W_n)\big)_{i,j} = \big(\nabla^2_{\theta} \tilde\Lambda(\theta)\big)_{i,j} \\
    = \frac{\int_a^b \tilde\varphi_i(x)\exp\!\big(\theta^\tr \tilde\varphi(x)\big)\de x \int_a^b \tilde\varphi_j(x)\exp\!\big(\theta^\tr \tilde\varphi(x)\big)\de x}{\Big(\int_a^b\exp\!\big(\theta^\tr \tilde\varphi(x)\big)\de x\Big)^2}
    - \frac{\int_a^b \tilde\varphi_i(x)\tilde\varphi_j(x)\exp\!\big(\theta^\tr \tilde\varphi(x)\big)\de x}{\Big(\int_a^b\exp\!\big(\theta^\tr \tilde\varphi(x)\big)\de x\Big)^2}.
\end{gather*}

To satisfy conditions (ii) and (iii), we need to prove that $\H(\theta)$ is strictly negative definite. We consider a random variable $\mathbf Y$ on $[a,b]$ with density
\begin{equation*}
    f(x) = \exp\!\big(\theta^\tr \tilde\varphi(x)\big) \Big/ \int_a^b\exp\!\big(\theta^\tr \tilde\varphi(x)\big)\de x,\qquad\text{for }x\in[a,b].
\end{equation*}
Then, one can show that the covariance matrix of $\tilde\varphi(\mathbf Y)$ is equal to
\begin{equation*}
    \Cov\big(\tilde\varphi(\mathbf Y)\big) = -\H(\theta),
\end{equation*}
so that $\H(\theta)$ is negative semi-definite. We conclude showing that $u^\tr H(\theta)u=0$ if and only if $\R^{m+k}\ni u = 0$. If
\begin{equation*}
    0=u^\tr H(\theta)u= - u^\tr \Cov\big(\tilde\varphi(\mathbf Y)\big)u=\var(u^\tr\tilde\varphi(\mathbf Y))
\end{equation*}
then $u^\tr\tilde\varphi(\mathbf Y)=c$ almost surely for some constant $c$. This implies that $u^\tr\tilde\varphi(x)=c$ for all $x\in[a,b]$. Since $0$ is the only constant in $\mathbb G^\times$, we have $u^\tr\tilde\varphi(x)=0$ for all $x\in[a,b]$ and so $u=0$.
\end{proof}


\begin{proposition}\label{prop:simple_MLE_cond_1}
    The functional $\Lambda$ as in \autoref{eq:Lambda_simple_MLE} satisfies the hypothesis of \autoref{lem:cond_1}, and so Condition 1 of \autoref{thm:approx_error} is satisfied.
\end{proposition}

\begin{proof}\mbox{}

We have
\begin{gather*}
    \Lambda(h_1 + \alpha h_2) = \E[h_1(\mathbf X)] + \alpha\E[h_2(\mathbf X)] - \log \int_a^b \exp\big(h_1(x) + \alpha h_2(x)\big)\de x\qquad\text{and} \\
    \frac{\de^2}{\de\alpha^2}\Lambda(h_1 + \alpha h_2) \\
    = \bigg(\frac{\int_a^b h_2(x)\exp\big(h_1(x) + \alpha h_2(x)\big)\de x}{\int_a^b \exp\big(h_1(x) + \alpha h_2(x)\big)\de x}\bigg)^2 - \frac{\int_a^b h_2(x)^2\exp\big(h_1(x) + \alpha h_2(x)\big)\de x}{\int_a^b \exp\big(h_1(x) + \alpha h_2(x)\big)\de x}.
\end{gather*}

We define the random variable $\mathbf X_\alpha$ that has density
\begin{equation*}
    f_{\mathbf X_\alpha}(x) = \exp\big(h_1(x) + \alpha h_2(x)\big) \Big/ \int_a^b \exp\big(h_1(x) + \alpha h_2(x)\big)\de x.
\end{equation*}
Then,
\begin{equation}\label{eq:second_Lambda}
    \frac{\de^2}{\de\alpha^2}\Lambda(h_1 + \alpha h_2) = -\var\big(h_2(\mathbf X_\alpha) \big).
\end{equation}

Assume $0\leq\alpha\leq1$, $\|h_1\|_\infty\leq C$, and $\|h_2\|_\infty\leq B$, for $B>0$ and $C>0$. Then\linebreak $\|h_1 + \alpha h_2\|_\infty\leq B+C$. It follows that
\begin{equation*}
    f_{\mathbf X_\alpha}(x)=\frac{\exp\big(h_1(x) + \alpha h_2(x)\big)}{\int_a^b \exp\big(h_1(x) + \alpha h_2(x)\big)\de x} \leq \frac{\exp\big(B+C\big)}{\int_a^b \exp\big(-B-C\big)\de x} = \frac{\exp\big(2B+2C\big)}{b-a} = \frac{M_1}{b-a}
\end{equation*}
and
\begin{equation*}
    f_{\mathbf X_\alpha}(x)=\frac{\exp\big(h_1(x) + \alpha h_2(x)\big)}{\int_a^b \exp\big(h_1(x) + \alpha h_2(x)\big)\de x} \geq \frac{\exp\big(-B-C\big)}{\int_a^b \exp\big(B+C\big)\de x} = \frac{\exp\big(-2B-2C\big)}{b-a} = \frac{M_2}{b-a},
\end{equation*}
with
\begin{equation*}
    M_1 = \exp\big(2B+2C\big)\qquad\text{and}\qquad M_2=\exp\big(-2B-2C\big).
\end{equation*}

For fixed $c\in[a,b]$, we have
\begin{equation*}
    \frac{M_2}{b-a}\int_a^b\big(h_2(x)-c\big)^2\de x \leq \int_a^b\big(h_2(x)-c\big)^2 f_{\mathbf X_\alpha}(x)\de x \leq \frac{M_1}{b-a}\int_a^b\big(h_2(x)-c\big)^2\de x
\end{equation*}
and, taking the infima over $c$, we have
\begin{equation*}\label{eq:XalphaU}
    M_2 \var\big(h_2(\mathbf U)\big) \leq \var\big(h_2(\mathbf X_\alpha)\big) \leq M_1 \var\big(h_2(\mathbf U)\big)
\end{equation*}
where $\mathbf U$ has uniform distribution over $[a,b]$. 

Since
\begin{equation*}
    \E[h_2(\mathbf U)] = \frac{1}{b-a} \int_a^b h_2(x) \de x = 0,
\end{equation*}
we have that
\begin{equation*}
    \var\big(h_2(\mathbf U)\big) = \E\big[h_2(\mathbf U)^2\big] = \frac{1}{b-a}\int_a^b h_2(x)^2 \de x = \frac{1}{b-a}\|h_2\|_2^2.
\end{equation*}

Finally,
\begin{equation*}
     \frac{M_2}{b-a}\|h_2\|_2^2 \leq \var\big(h_2(\mathbf X_\alpha)\big) \leq \frac{M_1}{b-a}\|h_2\|_2^2
\end{equation*}
or, equivalently,
\begin{equation*}
     -\frac{M_1}{b-a}\|h_2\|_2^2 \leq \frac{\de^2}{\de\alpha^2}\Lambda(h_1 + \alpha h_2) \leq -\frac{M_2}{b-a}\|h_2\|_2^2.
\end{equation*}

Since the hypothesis is satisfied, we can use \autoref{lem:cond_1} and conclude that in this setting Condition 1 of \autoref{thm:approx_error} is satisfied.
\end{proof}

\begin{proposition}
    The function $l$ as in \autoref{eq:ell_simple_MLE} satisfies the hypothesis of \autoref{lem:cond_2}, and so Condition 1 of \autoref{thm:est_error} is satisfied.
\end{proposition}

\begin{proof}\mbox{}

We recall that
\begin{equation*}
    \dot l[\bar\eta_n; g] = \frac{\de}{\de\alpha}l(\bar\eta_n+\alpha g)\bigg|_{\alpha=0^+}.
\end{equation*}

In our case,
\begin{equation*}
    l(\eta, \mathbf X) = \eta(\mathbf X) - \ln\int_a^b\exp(\eta(x))\de x.
\end{equation*}
Then,
\begin{gather*}
    l(\bar\eta_n+\alpha g) = \bar\eta_n(\mathbf X)+\alpha g(\mathbf X) - \ln\int_a^b\exp(\bar\eta_n(x)+\alpha g(x))\de x, \\
    \frac{\de}{\de\alpha}l(\bar\eta_n+\alpha g) = g(\mathbf X) - \frac{\int_a^bg(x)\exp(\bar\eta_n(x)+\alpha g(x))\de x}{\int_a^b\exp(\bar\eta_n(x)+\alpha g(x))\de x}, \text{ and}\\
    \dot l[\bar\eta_n; g]=g(\mathbf X) - \frac{\int_a^bg(x)\exp(\bar\eta_n(x))\de x}{\int_a^b\exp(\bar\eta_n(x))\de x}.
\end{gather*}
It follows that
\begin{equation*}
    \var\big(\dot l[\bar\eta_n; g]\big) = \var\big(g(\mathbf X)\big) = \E[g(\mathbf X)^2] - \E[g(\mathbf X)]^2 \leq \E[g(\mathbf X)^2] \leq C_2^*\|g\|_2^2,
\end{equation*}
since $f_0$ is bounded away from infinity.

Since the hypothesis is satisfied, we can use \autoref{lem:cond_2} and conclude that in this setting Condition 1 of \autoref{thm:est_error} is satisfied.
\end{proof}

\begin{proposition}\label{prop:simple_MLE_cond_3}
    The function $\ell$ as in \autoref{eq:ell_simple_MLE} satisfies the hypotheses of \autoref{lem:cond_3}, and so Condition 2 of \autoref{thm:est_error} is satisfied.
\end{proposition}

\begin{proof}\mbox{}

Since Condition 1 of \autoref{thm:approx_error} is satisfied we can apply \autoref{thm:approx_error} and obtain that $\|\bar\eta_n\|_\infty$ is bounded and so Condition (i) of \autoref{lem:cond_3} is satisfied.

We have that
\begin{gather*}
    \ell(\bar\eta_n + \alpha g) = \frac{1}{n}\sum_{i=1}^n \bigg(\bar\eta_n(\mathbf X_i) + \alpha g(\mathbf X_i) - \ln\int_a^b\exp(\bar\eta_n(x) + \alpha g(x))\de x\bigg)\qquad\text{and}\\
    \frac{\de^2}{\de\alpha^2}\ell(\bar\eta_n + \alpha g) = \\
    -\frac{1}{n}\sum_{i=1}^n \bigg(\frac{\int_a^bg(x)^2\exp(\bar\eta_n(x) + \alpha g(x))\de x}{\int_a^b\exp(\bar\eta_n(x) + \alpha g(x))\de x} - \Big(\frac{\int_a^bg(x)\exp(\bar\eta_n(x) + \alpha g(x))\de x}{\int_a^b\exp(\bar\eta_n(x) + \alpha g(x))\de x}\Big)^2\bigg) \\
    = -\var\big(g(\bar{\mathbf X}_\alpha) \big) \overset{\star}= \frac{\de^2}{\de\alpha^2}\Lambda(\bar\eta_n + \alpha g)
\end{gather*}
where the random variable $\mathbf X_\alpha$ is as in the proof of \autoref{prop:simple_MLE_cond_1} with $h_1=\bar\eta_n$ and $h_2=g$ and $\overset{\star}=$ is justified using \autoref{eq:second_Lambda}.

It follows that $\ell(\bar\eta_n + \alpha g)$ is twice differentiable. In addition, we have
\begin{equation*}
    \frac{\de^2}{\de\alpha^2}\ell(\bar\eta_n + \alpha g) = \frac{\de^2}{\de\alpha^2}\Lambda(\bar\eta_n + \alpha g) \leq -M_2 \|g\|_2^2,\qquad\text{ for }0\leq\alpha\leq1,
\end{equation*}
since, by \autoref{prop:simple_MLE_cond_1}, $\Lambda$ satisfies the hypothesis of \autoref{lem:cond_1}.

Since the hypotheses are satisfied, we can use \autoref{lem:cond_3} and conclude that in this setting Condition 2 of \autoref{thm:est_error} is satisfied.
\end{proof}