\subsection{Generalized Score matching}
In this section, we present a generalized version of the simple score matching estimator of the density. This will allow us to remove the assumption that $f_0(a) = f_0(b) = 0$. The method was first introduced by Yu et al. in \cite{gen_sm}.

We consider a continuous random variable $\mathbf X$, that takes values in $[a,b]$, with unknown density function $f_0\in W^{p+1}$. As before, we define the score $\psi_0$ of $f_0$ as
\begin{equation*}
    \psi_0(x) = \frac{\de}{\de x}\ln f_0(x), \qquad\text{with }x\in[a,b],
\end{equation*}
where $\psi_0\in W^p$.

We define the functions
\begin{equation*}
    \phi(x) = \min\!\big\{x-a;\ b-x;\ 1\big\}\qquad\text{and}\qquad \kappa(x) = x^\beta
\end{equation*}
with $\beta>1$.

As in the simple score matching method, we want to estimate $\psi_0$ and then use \autoref{eq:psi_to_f} to obtain an estimate of $f_0$.

Let $\psi\in W^p$. We want to minimize
\begin{gather*}
    J(\psi) = \frac{1}{2} \E \Big[\kappa\big(\phi(\mathbf X)\big)(\psi(\mathbf X) - \psi_0(\mathbf X))^2\Big] \\
    = \frac{1}{2}\int_a^b \kappa\big(\phi(x)\big)\psi(x)^2 f_0(x)\de x - \int_a^b \kappa\big(\phi(x)\big)\psi(x)\psi_0(x)f_0(x)\de x \\
    + \frac{1}{2}\int_a^b \kappa\big(\phi(x)\big)\psi_0(x)^2f_0(x)\de x.
\end{gather*}

The third term does not depend on $\psi$ so we can remove it, as it is a constant. The second term can be rewritten as
\begin{gather*}
    \int_a^b \kappa\big(\phi(x)\big)\psi(x)\psi_0(x)f_0(x)\de x = \int_a^b \kappa\big(\phi(x)\big)\psi(x)f_0'(x)\de x = \\
    \big[\kappa\big(\phi(x)\big)\psi(x)f_0(x)\big]_a^b - \int_a^b \kappa\big(\phi(x)\big)\psi'(x)f_0(x)\de x - \int_a^b \phi'(x)\kappa'\big(\phi(x)\big)\psi(x)f_0(x)\de x \\
    = - \int_a^b \kappa\big(\phi(x)\big)\psi'(x)f_0(x)\de x - \int_a^b \phi'(x)\kappa'\big(\phi(x)\big)\psi(x)f_0(x)\de x
\end{gather*}
where the first equality uses the fact that
\begin{equation*}
    f_0(x)\psi_0(x) = f_0(x)\frac{\de}{\de x}\ln f_0(x) = f_0(x)\frac{f_0'(x)}{f_0(x)} = f_0'(x),
\end{equation*}
the second equality is by integration by parts, and the third equality uses the fact that $\kappa\big(\phi(a)\big) = \kappa\big(\phi(b)\big) = 0$. In addition, we multiply $J(\psi)$ by $-1$ in order to turn a loss function into a utility function.

Then, we define $\Lambda$ as
\begin{gather}
    \Lambda(\psi) = - \int_a^b \kappa\big(\phi(x)\big)\psi'(x)f_0(x)\de x - \int_a^b \phi'(x)\kappa'\big(\phi(x)\big)\psi(x)f_0(x)\de x \notag \\
    -\frac{1}{2}\int_a^b \kappa\big(\phi(x)\big)\psi(x)^2 f_0(x)\de x \notag \\
    = \E\bigg[\!- \kappa\big(\phi(\mathbf X)\big)\psi'(\mathbf X) - \phi'(\mathbf X)\kappa'\big(\phi(\mathbf X)\big)\psi(\mathbf X) - \frac{1}{2}\kappa\big(\phi(\mathbf X)\big)\psi(\mathbf X)^2\bigg], \label{eq:Lambda_gen_SM}
\end{gather}
so that $\ell$ can be defined as
\begin{gather}
    \ell(\psi; \mathbf X_1 , \ldots, \mathbf X_n) = \frac{1}{n}\sum_{i=1}^n l(\psi; \mathbf X_i)
    \notag \\
    = \frac{1}{n}\sum_{i=1}^n \bigg(\!- \kappa\big(\phi(\mathbf X_i)\big)\psi'(\mathbf X_i) - \phi'(\mathbf X_i)\kappa'\big(\phi(\mathbf X_i)\big)\psi(\mathbf X_i) - \frac{1}{2}\kappa\big(\phi(\mathbf X_i)\big)\psi(\mathbf X_i)^2\bigg).\label{eq:ell_gen_SM}
\end{gather}

As for the Simple Score Matching method, we show that $\ell$ is a valid utility function under the assumption that $\mathbf X$ has the whole interval $[a,b]$ as support.

\begin{proposition}
    If $\mathrm{supp}(\mathbf X) = [a,b]$, 
    then the function $\ell$ as in \autoref{eq:ell_gen_SM} is a valid utility function.
\end{proposition}

\begin{proof}\mbox{}

We verify the hypotheses of \autoref{prop:validity} so we can conclude that $\ell$ is valid.

Condition (i) is satisfied by construction.

We define $\tilde\ell$ and $\tilde\Lambda$ as in \autoref{prop:validity}. We have
\begin{gather*}
    \tilde\ell(\theta; \mathbf X_1 , \ldots, \mathbf X_n) \\
    = \frac{1}{n}\sum_{v=1}^n \bigg(\!- \kappa\big(\phi(\mathbf X_v)\big)\theta^\tr \varphi'(\mathbf X_v) - \phi'(\mathbf X_v)\kappa'\big(\phi(\mathbf X_v)\big)\theta^\tr \varphi(\mathbf X_v) - \frac{1}{2}\kappa\big(\phi(\mathbf X_v)\big)\big(\theta^\tr \varphi(\mathbf X_v)\big)^2\bigg)
\end{gather*}
and
\begin{equation*}
    \Lambda(\theta) = \E\bigg[\!- \kappa\big(\phi(\mathbf X)\big)\theta^\tr \varphi'(\mathbf X) - \phi'(\mathbf X)\kappa'\big(\phi(\mathbf X)\big)\theta^\tr \varphi(\mathbf X) - \frac{1}{2}\kappa\big(\phi(\mathbf X)\big)\big(\theta^\tr \varphi(\mathbf X)\big)^2\bigg],
\end{equation*}
for $\theta\in\R^{m+k+1}$ and $\{\varphi_0,\ldots, \varphi_{m+k}\}$ basis of $\mathbb G$.

In order to verify condition (ii), we differentiate $\ell$ twice, we find
\begin{equation*}
    \big(\nabla^2_{\theta} \tilde\ell(\theta;\mathbf W_1,\ldots,\mathbf W_n)\big)_{i,j} = - \frac{1}{n}\sum_{v=1}^n \kappa\big(\phi(\mathbf X_v)\big)\varphi_i(\mathbf X_v)\varphi_j(\mathbf X_v),
\end{equation*}
so that
\begin{equation*}
    \nabla^2_{\theta} \tilde\ell(\theta;\mathbf W_1,\ldots,\mathbf W_n) = -\frac{1}{n}\sum_{v=1}^n \kappa\big(\phi(\mathbf X_v)\big)\varphi(\mathbf X_v)\varphi(\mathbf X_v)^\tr.
\end{equation*}
Since $\varphi(\mathbf X_v)\varphi(\mathbf X_v)^\tr$ is positive semi-definite for all $v\in{1,\ldots,n}$, $\nabla^2_{\theta} \tilde\ell(\theta;\mathbf W_1,\ldots,\mathbf W_n)$ is negative semi-definite for all $\theta\in\R^{m+k+1}$.

In order to verify condition (iii), we differentiate $\Lambda$ twice, we find
\begin{equation*}
    \big(\nabla^2_{\theta} \tilde\Lambda(\theta)\big)_{i,j} = \E\Big[\kappa\big(\phi(\mathbf X)\big)\varphi_i(\mathbf X)\varphi_j(\mathbf X)\Big],
\end{equation*}
so that
\begin{equation*}
    \nabla^2_{\theta} \tilde\Lambda(\theta) = \E\big[\kappa\big(\phi(\mathbf X)\big)\varphi(\mathbf X)\varphi(\mathbf X)^\tr\big].
\end{equation*}

Take $u\in\R^{m+k+1}$. Then
\begin{equation*}
    u^\tr \big(\nabla^2_{\theta} \tilde\Lambda(\theta)\big)u = \E\big[\kappa\big(\phi(\mathbf X)\big)\big(u^\tr\varphi(\mathbf X)\big)^2\big] \geq 0.
\end{equation*}
If $\E\big[\kappa\big(\phi(\mathbf X)\big)\big(u^\tr\varphi(\mathbf X)\big)^2\big]=0$, then $u^\tr\varphi(\mathbf X) = 0$ almost surely, since $\kappa\big(\phi(x)\big)$ is strictly positive for $x\in(a,b)$. Since $\mathrm{supp}(\mathbf X) = [a,b]$ by hypothesis, using \autoref{lem:support}, we have $u^\tr\tilde\varphi(x)=0$ for all $x\in[a,b]$ and so $u=0$.
\end{proof}

Unfortunately, similarly to the simple score matching method, $\Lambda$ does not satisfy the hypotheses of the probabilistic bounds. For example, if we try again to apply \autoref{lem:cond_1}, we obtain
\begin{gather*}
    \frac{\de^2}{\de\alpha^2}\Lambda(h_1+\alpha h_2) = -\E\big[\kappa\big(\phi(\mathbf X)\big)h_2(\mathbf X)^2\big] = -\int_a^b \kappa\big(\phi(x)\big)h_2(x)^2f_0(x)\de x.
\end{gather*}

Since $\kappa\big(\phi(a)\big) = \kappa\big(\phi(b)\big) = 0$, in general we cannot find a constant $M_2>0$ such that
\begin{equation*}
    \frac{\de^2}{\de\alpha^2}\Lambda(h_1+\alpha h_2) \leq -M_2\|h_2\|_2^2, 
    \quad 0 \leq \alpha \leq 1.
\end{equation*}