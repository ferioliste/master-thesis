\section{Implementation}\label{ch:chapter3}
In this chapter, we outline how the five methods for probability density estimation can be implemented on a computer. The implementation builds on the work of Kooperberg and Stone \cite{kooperberg1992}, who describe an algorithm for the Simple Maximum Likelihood estimator in the case $m=3$ and without a penalty term. Here, we extend their approach to arbitrary spline degree $m$, include a penalty term, and follow the same general strategy to implement the other four methods.

For simplicity, we use uniformly spaced knots on $[a,b]$, although \cite{kooperberg1992} also presents some interesting heuristics to place the knots more optimally. After describing the implementation of the algorithms, we present two cross-validation routines to select the penalty parameter $\lambda_n$. Finally, we comment on the time complexity of the algorithms.

All algorithms were implemented in Python and collected into a library\footnote{The library is available at \href{https://github.com/ferioliste/spline-density-estimator}{\underline{https://github.com/ferioliste/spline-density-estimator}}}.

\subsection{Representing splines}
The first point we have to deal with is which basis to use in order to represent the splines in $\mathbb G$ in a way that is convenient for numerical computations. We define the truncated power basis functions $\varphi_i$ by
\begin{equation*}
    \varphi_i(x) = x^i,\qquad i=0,\ldots,m,
\end{equation*}
and
\begin{equation*}
    \varphi_{m+j}(x) = (x-t_j)_+^m,\qquad j=1,\ldots,k,
\end{equation*}
where $(u)_+ = \max\{u,0\}$.

By construction, the family $\{\varphi_0,\ldots,\varphi_{m+k}\} \subset \mathbb G$ contains $m+k+1$ linearly independent functions and thus forms a basis of $\mathbb G$. In particular, for any spline $g\in\mathbb G$ there exists a unique coefficient vector $\theta\in\R^{m+k+1}$ such that
\begin{equation}\label{eq:representation}
    g(x) = \theta^\tr \varphi(x) = \sum_{i=0}^{m+k} \theta_i\varphi_i(x)
    = \sum_{i=0}^m \theta_i x^i + \sum_{j=1}^k \theta_{m+j}(x-t_j)_+^m,
    \qquad x\in[a,b],
\end{equation}
where $\varphi(x)\in\R^{m+k+1}$ is such that $\big(\varphi(x)\big)_i = \varphi_i(x)$ for $i=0,\ldots,m+k$.

In our implementation, we estimate the coefficient vector of $\hat\eta_n$.
\subsection{Optimization algorithm}
We assume to have available $n$ iid realizations of $\mathbf X$, denoted as $\mathbf X_1,\dots,\mathbf X_n$.

We recall that all the estimators that we presented are in the form
\begin{equation*}
    \hat\eta_n = \argmax_{h\in \mathbb G} \p\ell(h; \mathbf X_1 , \ldots, \mathbf X_n).
\end{equation*}

Using the representation \ref{eq:representation}, we can reformulate the problem as
\begin{equation*}
    \hat \theta = \argmax_{\theta\in\R^{m+k+1}} \mathcal L(\theta)\qquad\text{with}\qquad\mathcal L(\theta) = \p\ell\Big(\theta^\tr \varphi(x); \mathbf X_1 , \ldots, \mathbf X_n\Big)
\end{equation*}
and $\hat\eta_n = \hat\theta^\tr \varphi(x)$.

In order to maximize $\mathcal L$ we use the Newton-Raphson method. We start with an initial guess $\theta^{(0)}$ and we produce a sequence of coefficient vectors using the formula
\begin{equation*}
    \theta^{(i+1)} = \theta^{(i)} + \zeta\big[\H(\theta^{(i)})\big]^{-1} \S(\theta^{(i)})
\end{equation*}
where $0<\zeta\leq 1$, $\S(\theta^{(i)}) \in\R^{m+k+1}$ is the gradient of $\mathcal L$ at $\theta^{(i)}$ and $\H(\theta^{(i)}) \in\R^{(m+k+1)\times (m+k+1)}$ is the Hessian of $\mathcal L$ at $\theta^{(i)}$.

We stop the iterations when
\begin{equation*}
    \frac{\mathcal L(\theta^{(i+1)}) - \mathcal L(\theta^{(i)})}{\max\{|\mathcal L(\theta^{(i)}|,\, 1\}} \leq \varepsilon
\end{equation*}
for $\varepsilon > 0$.

For the different methods, we will verify that $\mathcal L$ satisfies the hypotheses that ensure that the Newton-Raphson method converges to a maximizer. In particular, we will check that $\mathcal L$ is twice continuously differentiable on $\R^{m+k+1}$ and that $\H(\theta)$ is non-singular in a neighborhood of the solution.

\begin{algorithm}[H]
\caption{Newton-Raphson maximization of $\mathcal L$}
\label{alg:newton_max}
\begin{algorithmic}[1]
\REQUIRE Data $\mathbf X_1,\ldots,\mathbf X_n$, initial value $\theta^{(0)}\in\R^{m+k+1}$, step size $\zeta\in(0,1]$, tolerance $\varepsilon>0$, maximum number of iterations $I_{\max}$.
\STATE Set $i\gets 0$.
\STATE Set $r_0 = 1 + \varepsilon$
\WHILE{$r_i \geq \varepsilon$ \AND $i < I_{\max}$}
    \STATE Compute the gradient $\S(\theta^{(i)})$.
    \STATE Compute the Hessian $\H(\theta^{(i)})$.
    \STATE Solve the linear system $\H(\theta^{(i)})\,d^{(i)} = \S(\theta^{(i)})$ for $d^{(i)}\in\R^{m+k+1}$.
    \STATE Set $\theta^{(i+1)} \gets \theta^{(i)} + \zeta\, d^{(i)}$.
    \STATE Set $i\gets i+1$.
    \STATE Set $\displaystyle r_i \gets \frac{\mathcal L(\theta^{(i+1)}) - \mathcal L(\theta^{(i)})}{\max\{|\mathcal L(\theta^{(i)})|,\, 1\}}.$
\ENDWHILE
\RETURN $\theta^{(i)}$
\end{algorithmic}
\end{algorithm}
\subsection{Penalty term}\label{sec:penalty}
In this section, we show how we can compute the penalty term of the penalized utility functions, as it is the same in every method. We recall that
\begin{equation*}
    J_q(g) = \int_a^b \big(g^{(q)}(x)\big)^2 \de x = \|g^{(q)}\|_2^2
\end{equation*}

Let $\theta$ be the coefficient vector of $g$. Then we have
\begin{equation*}
    J_q(g) = \int_a^b \Big(\sum_{i=0}^{m+k} \theta_i\varphi_i^{(q)}(x)\Big)^2 \de x = \sum_{i=0}^{m+k}\sum_{j=0}^{m+k} \theta_i \theta_j \int_a^b\varphi_i^{(q)}(x)\varphi_j^{(q)}(x) \de x = \theta^\tr G_q \theta
\end{equation*}
where $G_q\in\R^{(k+m+1)\times(k+m+1)}$ is the symmetric positive semi-definite matrix with entries
\begin{equation}
    (G_q)_{i,j} = \int_a^b \varphi_i^{(q)}(x)\varphi_j^{(q)}(x)\,\de x,
    \qquad\substack{\displaystyle i=0,\ldots,m+k,\\ \displaystyle j=0,\ldots,m+k.}
\end{equation}

In the truncated power basis, $q$ derivatives are identically zero. More precisely, for $0\leq i<q$,
\begin{equation*}
    \varphi_i^{(q)}\equiv0
\end{equation*}
For $q\leq i \leq m$,
\begin{equation*}
    \varphi_i^{(q)}(x) = \frac{i!}{(i-q)!}\,x^{i-q}.
\end{equation*}
For $j=1,\ldots,k$,
\begin{equation*}
    \varphi_{m+j}^{(q)}(x)
    = \frac{m!}{(m-q)!}(x-t_j)_+^{m-q},
\end{equation*}
with the convention that $(x-t_j)_+^{0}=\I\{x>t_j\}$.

It follows that the first $q$ rows and columns of $G_q$ are filled with $0$. In addition, every entry of $G_q$ can be calculated exactly by integrating polynomials. For example, take $i\in\{q,\ldots,m\}$ and $j\in\{1,\ldots,k\}$, then
\begin{gather*}
    (G_q)_{i,m+j} = \frac{i!}{(i-q)!}\frac{m!}{(m-q)!} \int_a^b x^{i-q} (x-t_j)_+^{m-q} \de x \\
    = \frac{i!}{(i-q)!}\frac{m!}{(m-q)!} \int_{t_j}^b x^{i-q} (x-t_j)^{m-q} \de x \\
    = \frac{i!}{(i-q)!}\frac{m!}{(m-q)!} \int_{t_j}^b \sum_{v=0}^{m-q} \tbinom{m-q}{v} x^{i+m-2q-v} (-t_j)^v \de x \\
    = \frac{i!}{(i-q)!}\frac{m!}{(m-q)!}\sum_{v=0}^{m-q} \tbinom{m-q}{v}(-t_j)^v\int_{t_j}^b x^{i+m-2q-v} \de x \\
    = \frac{i!}{(i-q)!}\frac{m!}{(m-q)!}\sum_{v=0}^{m-q} \tbinom{m-q}{v}(-t_j)^v \frac{b^{i+m-2q-v+1} - t_j^{i+m-2q-v+1}}{i+m-2q-v+1}.
\end{gather*}

With a bit more work one can calculate also $(G_q)_{i,j}$ for $i,j\in\{q,\ldots,m\}$ and $(G_q)_{m+i,m+j}$ for $i,j\in\{1,\ldots,k\}$. 

It's important to notice that $G_q$ depends only on the basis and does not change between iterations. Therefore, it can be computed once and stored, so that the penalty evaluation reduces to evaluating the quadratic form $\theta^\tr G_q \theta$.

The matrix $G_q$ can be used to evaluate all integrals of the form
\begin{equation*}
    \int_a^b g^{(q)}(x)h^{(q)}(x)\de x.
\end{equation*}
Let $\theta$ and $\theta^*$ be the coefficient vectors of $g$ and $h$, respectively. Then
\begin{equation*}
    \int_a^b g^{(q)}(x)h^{(q)}(x)\de x = \theta^\tr G_q \theta^*.
\end{equation*}

In particular, we have that
\begin{equation}\label{eq:normal_integral}
    \int_a^b g(x)\de x = \int_a^b 1\cdot g(x)\de x = e_0^\tr G_0 \theta,
\end{equation}
where $e_0=(1,0,\ldots,0)^\tr\in\R^{m+k+1}$ is the first canonical basis vector, and
\begin{equation}\label{eq:squared_integral}
    \int_a^b g(x)^2\de x = \theta^\tr G_0 \theta,
\end{equation}
\subsection{Implementation of Simple Maximum Likelihood}
We recall that for the Simple Maximum Likelihood estimator, we have
\begin{equation*}
     \ell(\eta; \mathbf X_1 , \ldots, \mathbf X_n) = \frac{1}{n}\sum_{v=1}^n \bigg(\eta(\mathbf X_v) - \ln \int_a^b \exp \eta(x)\de x\bigg),
\end{equation*}
so that
\begin{equation*}
    \mathcal L(\theta) = \frac{1}{n}\sum_{v=1}^n \theta^\tr \varphi(\mathbf X_v) - \ln \int_a^b \exp\!\big(\theta^\tr \varphi(x)\big)\de x - \lambda_n\,\theta^\tr G_q \theta.
\end{equation*}

Instead of imposing
\begin{equation*}
    \int_a^b\theta^\tr \varphi(x)\de x = 0,
\end{equation*}
we use the condition
\begin{equation*}
    \theta_0 = 0,
\end{equation*}
that also guarantees a one-to-one correspondence between $\theta^\tr \varphi(x)$ and density.

For fixed $\theta$, we calculate $\mathcal L(\theta)$ and the gradient and the Hessian of $\mathcal L$ at $\theta$. 
First, we define $I^{(0)}\in\R$, $I^{(1)}\in\R^{m+k+1}$, and $I^{(2)}\in\R^{(m+k+1)\times(m+k+1)}$, as
\begin{gather*}
    I^{(0)} = \int_a^b\exp\!\big(\theta^\tr \varphi(x)\big)\de x, \\
    I^{(1)}_i = \int_a^b \varphi_i(x)\exp\!\big(\theta^\tr \varphi(x)\big)\de x,\qquad i=0,\ldots,m+k,\quad\text{and}\\
    I^{(2)}_{i,j} = \int_a^b \varphi_i(x)\varphi_j(x)\exp\!\big(\theta^\tr \varphi(x)\big)\de x,\qquad\substack{\displaystyle i=0,\ldots,m+k,\\ \displaystyle j=0,\ldots,m+k.}
\end{gather*}

For $\mathcal L(\theta)$, we have
\begin{equation*}
    \mathcal L(\theta) = \frac{1}{n}\sum_{v=1}^n \theta^\tr \varphi(\mathbf X_v) - \ln I^{(0)} - \lambda_n\,\theta^\tr G_q \theta.
\end{equation*}

For the gradient, we have
\begin{gather*}
    \big(\S(\theta)\big)_i = \frac{\partial}{\partial\theta_i} \mathcal L(\theta) = \frac{1}{n}\sum_{v=1}^n \varphi_i(\mathbf X_v) - \frac{\int_a^b \varphi_i(x)\exp\!\big(\theta^\tr \varphi(x)\big)\de x}{\int_a^b\exp\!\big(\theta^\tr \varphi(x)\big)\de x} - 2\lambda_n\big(G_q \theta\big)_i \\
    = \frac{1}{n}\sum_{v=1}^n \varphi_i(\mathbf X_v) - \frac{I^{(1)}_i}{I^{(0)}} - 2\lambda_n\big(G_q \theta\big)_i,
\end{gather*}
for $i=0,\ldots,m+k$.

For the Hessian, we have
\begin{gather*}
    \big(\H(\theta)\big)_{i,j} = \frac{\partial^2}{\partial\theta_i\partial\theta_j} \mathcal L(\theta) = \frac{\int_a^b \varphi_i(x)\exp\!\big(\theta^\tr \varphi(x)\big)\de x \int_a^b \varphi_j(x)\exp\!\big(\theta^\tr \varphi(x)\big)\de x}{\Big(\int_a^b\exp\!\big(\theta^\tr \varphi(x)\big)\de x\Big)^2} \\
    - \frac{\int_a^b \varphi_i(x)\varphi_j(x)\exp\!\big(\theta^\tr \varphi(x)\big)\de x}{\Big(\int_a^b\exp\!\big(\theta^\tr \varphi(x)\big)\de x\Big)^2} - 2\lambda_n\big(G_q\big)_{i,j} = \frac{I^{(1)}_i I^{(1)}_j -I^{(2)}_{i,j}}{\big(I^{(0)}\big)^2} - 2\lambda_n\big(G_q\big)_{i,j},
\end{gather*}
for $i=0,\ldots,m+k$ and $j=0,\ldots,m+k$.

We define the matrix $A\in\R^{(k+1)\times (2m+1)}$, with entries
\begin{equation*}
    A_{i,j} = \int_{t_i}^{t_{i+1}} x^{j}\exp\!\big(\theta^\tr \varphi(x)\big)\de x
\end{equation*}
for $i=0,\ldots,k$ and $j=0,\ldots,2m$. 

We approximate the entries of $A$ using Gauss-Legendre quadrature. Given a function $f:[-1,1]\rightarrow\R$, the integral on $f$ on $[-1,1]$ can be approximated using the $\alpha$-point Gauss-Legendre rule
\begin{equation*}
    \int_{-1}^{1} f(x)\de x \ \approx\; \sum_{i=1}^{\alpha} w_i f(x_i),
\end{equation*}
where the nodes $x_1,\ldots,x_\alpha$ are the roots of the Legendre polynomial of order $\alpha$, $P_\alpha$, and
\begin{equation*}
    w_i=\frac{2}{\big(1-x_i^2\big)\big(P_\alpha'(x_i)\big)^2}.
\end{equation*}
for $i=1,\ldots,\alpha$. 

The $\alpha$-point Gauss-Legendre quadrature rule is exact for all polynomials of degree at most $2\alpha-1$ and is very precise with all $f$ that can be well-approximated by polynomials. We can approximate integrals on any interval using a simple change of variable.

The entries of $A$ can be used to calculate $I^{(0)}$, $I^{(1)}$, and $I^{(2)}$. We give a few examples of calculations. First, we have
\begin{equation*}
    I^{(0)} = \int_a^b\exp\!\big(\theta^\tr \varphi(x)\big)\de x = \sum_{i=0}^k \int_{t_i}^{t_{i+1}}\exp\!\big(\theta^\tr \varphi(x)\big)\de x = \sum_{i=0}^k A_{i,0}.
\end{equation*}

For $j\in\{1,\ldots,k\}$,
\begin{gather*}
    I^{(1)}_{m+j} = \int_a^b (x-t_j)_+^m\exp\!\big(\theta^\tr \varphi(x)\big)\de x = \int_{t_j}^b (x-t_j)^m\exp\!\big(\theta^\tr \varphi(x)\big)\de x \\
    = \sum_{v=j}^k \int_{t_v}^{t_{v+1}} (x-t_j)^m\exp\!\big(\theta^\tr \varphi(x)\big)\de x = \sum_{v=j}^k \int_{t_v}^{t_{v+1}} \sum_{z=0}^{m} \tbinom{m}{z} x^{m-z} (-t_j)^z \exp\!\big(\theta^\tr \varphi(x)\big)\de x \\
    = \sum_{v=j}^k \sum_{z=0}^{m} \tbinom{m}{z}(-t_j)^z \int_{t_v}^{t_{v+1}} x^{m-z} \exp\!\big(\theta^\tr \varphi(x)\big)\de x = \sum_{v=j}^k \sum_{z=0}^{m} \tbinom{m}{z}(-t_j)^z A_{v,(m-z)}.
\end{gather*}

For $i\in\{1,\ldots,k\}$ and $j\in\{1,\ldots,k\}$ with $i\leq j$,
\begin{gather*}
    I^{(2)}_{m+i,m+j} = \int_a^b (x-t_i)_+^m(x-t_j)_+^m\exp\!\big(\theta^\tr \varphi(x)\big)\de x = \int_{t_j}^b (x-t_i)^m(x-t_j)^m\exp\!\big(\theta^\tr \varphi(x)\big)\de x \\
    = \sum_{v=j}^k \int_{t_v}^{t_{v+1}} (x-t_i)^m(x-t_j)^m\exp\!\big(\theta^\tr \varphi(x)\big)\de x \\
    = \sum_{v=j}^k \int_{t_v}^{t_{v+1}} \sum_{z=0}^{m}\sum_{y=0}^{m} \tbinom{m}{z}\tbinom{m}{y} x^{2m-z-y} (-t_i)^z (-t_j)^y \exp\!\big(\theta^\tr \varphi(x)\big)\de x \\
    = \sum_{v=j}^k\sum_{z=0}^{m}\sum_{y=0}^{m} \tbinom{m}{z}\tbinom{m}{y} (-t_i)^z (-t_j)^y \int_{t_v}^{t_{v+1}} x^{2m-z-y} \exp\!\big(\theta^\tr \varphi(x)\big)\de x \\
    = \sum_{v=j}^k\sum_{z=0}^{m}\sum_{y=0}^{m} \tbinom{m}{z}\tbinom{m}{y} (-t_i)^z (-t_j)^y A_{v,(2m-z-y)}.
\end{gather*}

With a bit more work one can cover all the other cases to calculate all the entries of $I^{(0)}$, $I^{(1)}$, and $I^{(2)}$.

Now, we verify that $\mathcal L$ verifies the hypotheses of \autoref{prop:newton_global_max}. First, we showed that $\mathcal L$ is twice differentiable at every $\theta\in\R^{m+k+1}$.

Let $\theta\in\R^{m+k+1}$. We consider a random variable $\mathbf Y$ on $[a,b]$ with density
\begin{equation*}
    f(x) = \exp\!\big(\theta^\tr \varphi(x)\big) \Big/ \int_a^b\exp\!\big(\theta^\tr \varphi(x)\big)\de x,\qquad\text{for }x\in[a,b].
\end{equation*}
Then, one can show that the covariance matrix of $\varphi(\mathbf Y)$ is equal to
\begin{equation*}
    \Cov\big(\varphi(\mathbf Y)\big)_{i,j} = \frac{I^{(2)}_{i,j} - I^{(1)}_i I^{(1)}_j}{\big(I^{(0)}\big)^2}
\end{equation*}
for $i=0,\ldots,m+k$ and $j=0,\ldots,m+k$.

Using this, the Hessian $\H(\theta)$ can be rewritten as
\begin{equation*}
    \H(\theta) = -\Cov\big(\varphi(\mathbf Y)\big) - 2\lambda_n G_q,
\end{equation*}
and therefore $\H(\theta)$ is negative semi-definite for all $\theta\in\R^{m+k+1}$, since both $\Cov\big(\varphi(\mathbf Y)\big)$ and $G_q$ are positive semi-definite.

We summarize the complete procedure in \autoref{alg:simple_ML_newton}.

\begin{algorithm}[H]
\caption{Algorithm for Simple Maximum Likelihood}
\label{alg:simple_ML_newton}
\begin{algorithmic}[1]
\REQUIRE Data $\mathbf X_1,\ldots,\mathbf X_n$, initial value $\theta^{(0)}\in\R^{m+k+1}$ such that $\theta^{(0)}_0=0$, step size $\zeta\in(0,1]$, tolerance $\varepsilon>0$, maximum number of iterations $I_{\max}$, quadrature order $\alpha$.
\STATE Compute $G_q$
\STATE Set $i\gets 0$.
\STATE Set $L_{\text{new}}\gets -\infty$.
\WHILE{$i<I_{\max}$}
    \STATE Approximate $A$ using the $\alpha$-point Gauss-Legendre quadrature.
    \STATE Compute $I^{(0)}$, $I^{(1)}$, and $I^{(2)}$.
    \STATE Set $L_{\text{old}} \gets L_{\text{new}}$.
    \STATE Compute $\mathcal L(\theta^{(i)})$.
    \STATE Set $L_{\text{new}} \gets \mathcal L(\theta^{(i)})$.
    \IF{i>0}
        \STATE Set $\displaystyle r \gets \frac{L_{\text{new}} - L_{\text{old}}}{\max\{|L_{\text{old}}|,\,1\}}$
        \IF{$r<\varepsilon$}
            \STATE \textbf{break}
        \ENDIF
    \ENDIF
    \STATE Compute the gradient $\S(\theta^{(i)})$.
    \STATE Compute the Hessian $\H(\theta^{(i)})$.
    \STATE Solve the linear system $\H(\theta^{(i)})\,d^{(i)} = \S(\theta^{(i)})$ for $d^{(i)}\in\R^{m+k+1}$, with $d^{(i)}_0=0$.
    \STATE Set $\theta^{(i+1)} \gets \theta^{(i)} + \zeta\, d^{(i)}$.
    \STATE Set $i\gets i+1$.
\ENDWHILE
\RETURN $\theta^{(i)}$
\end{algorithmic}
\end{algorithm}
\subsection{Implementation of Shifted Maximum Likelihood}
In order to implement the Shifted Maximum Likelihood method, we follow a very similar procedure to the Simple Maximum Likelihood method. We highlight only the main steps.

We recall that
\begin{equation*}
    \ell(\eta; \mathbf X_1 , \ldots, \mathbf X_n)
    = \frac{1}{n}\sum_{v=1}^n \bigg((1-\gamma)\eta(\mathbf X_v) + \frac{\gamma}{b-a}\int_a^b\eta(x)\de x - \ln \int_a^b \exp \eta(x)\de x\bigg),
\end{equation*}
so that, using \autoref{eq:normal_integral},
\begin{equation*}
    \mathcal L(\theta) = \frac{1-\gamma}{n}\sum_{v=1}^n \theta^\tr \varphi(\mathbf X_v) - \ln \int_a^b \exp\!\big(\theta^\tr \varphi(x)\big)\de x + \frac{\gamma}{b-a}\, e_0^\tr G_0 \theta - \lambda_n\,\theta^\tr G_q \theta.
\end{equation*}

We use again the condition $\theta_0 = 0$.

For fixed $\theta$, we calculate $\mathcal L(\theta)$ and the gradient and the Hessian of $\mathcal L$ at $\theta$. 
We define and approximate $I^{(0)}\in\R$, $I^{(1)}\in\R^{m+k+1}$, and $I^{(2)}\in\R^{(m+k+1)\times(m+k+1)}$, as in the implementation of the Simple Maximum Likelihood method. Then we have
\begin{gather*}
    \mathcal L(\theta) = \frac{1-\gamma}{n}\sum_{v=1}^n \theta^\tr \varphi(\mathbf X_v) - \ln I^{(0)} + \frac{\gamma}{b-a}\, e_0^\tr G_0 \theta - \lambda_n\,\theta^\tr G_q \theta, \\
    \big(\S(\theta)\big)_i = \frac{\partial}{\partial\theta_i} \mathcal L(\theta) = \frac{1-\gamma}{n}\sum_{v=1}^n \varphi_i(\mathbf X_v) - \frac{I^{(1)}_i}{I^{(0)}} + \frac{\gamma}{b-a}\big(G_0\big)_{0,i} - 2\lambda_n\big(G_q \theta\big)_i,
\end{gather*}
for $i=0,\ldots,m+k$, and
\begin{equation*}
    \big(\H(\theta)\big)_{i,j} = \frac{\partial^2}{\partial\theta_i\partial\theta_j} \mathcal L(\theta) = \frac{I^{(1)}_i I^{(1)}_j -I^{(2)}_{i,j}}{\big(I^{(0)}\big)^2} - 2\lambda_n\big(G_q\big)_{i,j}
\end{equation*}
for $i=0,\ldots,m+k$ and $j=0,\ldots,m+k$.

Also in this case, $\mathcal L$ verifies the hypotheses of \autoref{prop:newton_global_max}. First, we showed that $\mathcal L$ is twice differentiable at every $\theta\in\R^{m+k+1}$. In addition, for $\theta\in\R^{m+k+1}$, $\H(\theta)$ is the same as for the Simple Maximum Likelihood method and therefore it is negative semi-definite.
\subsection{Implementation of Simple Score Matching}
We recall that for the Simple Score Matching estimator, we have
\begin{equation*}
     \ell(\psi; \mathbf X_1 , \ldots, \mathbf X_n) = \frac{1}{n}\sum_{v=1}^n \bigg(\!- \psi'(\mathbf X_v) - \frac{1}{2}\psi(\mathbf X_v)^2\bigg),
\end{equation*}
so that
\begin{equation*}
    \mathcal L(\theta) = \frac{1}{n}\sum_{v=1}^n \bigg(\!- \theta^\tr \varphi'(\mathbf X_v) - \frac{1}{2}\big(\theta^\tr \varphi(\mathbf X_v)\big)^2\bigg) - \lambda_n\,\theta^\tr G_q \theta,
\end{equation*}
where $\varphi'(x)\in\R^{m+k+1}$ is such that $\big(\varphi'(x)\big)_i = \varphi_i'(x)$ for $i=0,\ldots,m+k$.

We recall that $\varphi_0' \equiv 0$. For $i=1,\ldots,m$,
\begin{equation*}
    \varphi_i'(x) = i\,x^{i-1}.
\end{equation*}
For $j=1,\ldots,k$,
\begin{equation*}
    \varphi_{m+j}'(x)
    = m(x-t_j)_+^{m-1}.
\end{equation*}

For fixed $\theta$, we calculate the gradient and the Hessian of $\mathcal L$ at $\theta$. We have
\begin{equation*}
    \big(\S(\theta)\big)_i = \frac{\partial}{\partial\theta_i} \mathcal L(\theta) = \frac{1}{n}\sum_{v=1}^n \bigg(\!- \varphi_i'(\mathbf X_v) - \varphi_i(\mathbf X_v)\big(\theta^\tr \varphi(\mathbf X_v)\big)\bigg) - 2\lambda_n\big(G_q \theta\big)_i
\end{equation*}
for $i=0,\ldots,m+k$, and
\begin{gather*}
    \big(\H(\theta)\big)_{i,j} = \frac{\partial^2}{\partial\theta_i\partial\theta_j} \mathcal L(\theta) = \frac{1}{n}\sum_{v=1}^n \big(\!- \varphi_i(\mathbf X_v)\varphi_j(\mathbf X_v)\big) - 2\lambda_n\big(G_q\big)_{i,j},
\end{gather*}
for $i=0,\ldots,m+k$ and $j=0,\ldots,m+k$.

Now, we verify that $\mathcal L$ verifies the hypotheses of \autoref{prop:newton_global_max}. First, we showed that $\mathcal L$ is twice differentiable at every $\theta\in\R^{m+k+1}$.

Let $\theta\in\R^{m+k+1}$. We rewrite the Hessian $\H(\theta)$ as
\begin{equation*}
    \H(\theta) = -\bigg(\frac{1}{n}\sum_{v=1}^n \varphi(\mathbf X_v)\varphi(\mathbf X_v)^\tr + 2\lambda_n\,G_q\bigg).
\end{equation*}
Since $\varphi(\mathbf X_v)\varphi(\mathbf X_v)^\tr$ is positive semi-definite for all $v\in{1,\ldots,n}$ and $G_q$ is positive semi-definite, $\H(\theta)$ is negative semi-definite for all $\theta\in\R^{m+k+1}$.

We summarize the complete procedure in \autoref{alg:simple_SM_newton}.

\begin{algorithm}[H]
\caption{Algorithm for Simple Score Matching}
\label{alg:simple_SM_newton}
\begin{algorithmic}[1]
\REQUIRE Data $\mathbf X_1,\ldots,\mathbf X_n$, initial value $\theta^{(0)}\in\R^{m+k+1}$, step size $\zeta\in(0,1]$, tolerance $\varepsilon>0$, maximum number of iterations $I_{\max}$.
\STATE Set $i\gets 0$.
\STATE Set $L_{\text{new}}\gets -\infty$.
\WHILE{$i<I_{\max}$}
    \STATE Set $L_{\text{old}} \gets L_{\text{new}}$.
    \STATE Compute $\mathcal L(\theta^{(i)})$.
    \STATE Set $L_{\text{new}} \gets \mathcal L(\theta^{(i)})$.
    \IF{i>0}
        \STATE Set $\displaystyle r \gets \frac{L_{\text{new}} - L_{\text{old}}}{\max\{|L_{\text{old}}|,\,1\}}$
        \IF{$r<\varepsilon$}
            \STATE \textbf{break}
        \ENDIF
    \ENDIF
    \STATE Compute the gradient $\S(\theta^{(i)})$.
    \STATE Compute the Hessian $\H(\theta^{(i)})$.
    \STATE Solve the linear system $\H(\theta^{(i)})\,d^{(i)} = \S(\theta^{(i)})$ for $d^{(i)}\in\R^{m+k+1}$.
    \STATE Set $\theta^{(i+1)} \gets \theta^{(i)} + \zeta\, d^{(i)}$.
    \STATE Set $i\gets i+1$.
\ENDWHILE
\RETURN $\theta^{(i)}$
\end{algorithmic}
\end{algorithm}
\subsection{Implementation of Shifted Score Matching}
The implementation proceeds very similarly to the Simple Score Matching method. We recall that for the Shifted Score Matching estimator, we have
\begin{equation*}
     \ell(\psi; \mathbf X_1 , \ldots, \mathbf X_n)
    = \frac{1}{n}\sum_{v=1}^n \bigg(\!- (1-\gamma)\psi'(\mathbf X_v) - \frac{1}{2}(1-\gamma)\psi(\mathbf X_v)^2 - \frac{1}{2}\frac{\gamma}{b-a}\int_a^b \psi(x)^2 \de x\bigg)
\end{equation*}
so that
\begin{equation*}
    \mathcal L(\theta) = \frac{1}{n}\sum_{v=1}^n \bigg(\!- (1-\gamma)\theta^\tr \varphi'(\mathbf X_v) - \frac{1}{2}(1-\gamma)\big(\theta^\tr \varphi(\mathbf X_v)\big)^2\bigg) - \frac{1}{2}\frac{\gamma}{b-a}\,\theta^\tr G_0 \theta - \lambda_n\,\theta^\tr G_q \theta.
\end{equation*}

For fixed $\theta$, we calculate the gradient and the Hessian of $\mathcal L$ at $\theta$. We have
\begin{gather*}
    \big(\S(\theta)\big)_i = \frac{\partial}{\partial\theta_i} \mathcal L(\theta) \\
    = \frac{1}{n}\sum_{v=1}^n \bigg(\!- (1-\gamma)\varphi_i'(\mathbf X_v) - (1-\gamma)\varphi_i(\mathbf X_v)\big(\theta^\tr \varphi(\mathbf X_v)\big)\bigg) - \frac{1}{2}\frac{\gamma}{b-a}\big(G_0 \theta\big)_i - 2\lambda_n\big(G_q \theta\big)_i
\end{gather*}
for $i=0,\ldots,m+k$, and
\begin{gather*}
    \big(\H(\theta)\big)_{i,j} = \frac{\partial^2}{\partial\theta_i\partial\theta_j} \mathcal L(\theta) = \frac{1}{n}\sum_{v=1}^n \big(\!- (1-\gamma)\varphi_i(\mathbf X_v)\varphi_j(\mathbf X_v)\big) - \frac{1}{2}\frac{\gamma}{b-a}\big(G_0\big)_{i,j} - 2\lambda_n\big(G_q\big)_{i,j},
\end{gather*}
for $i=0,\ldots,m+k$ and $j=0,\ldots,m+k$.

Now, we verify that $\mathcal L$ verifies the hypotheses of \autoref{prop:newton_global_max}. First, we showed that $\mathcal L$ is twice differentiable at every $\theta\in\R^{m+k+1}$.

Let $\theta\in\R^{m+k+1}$. We rewrite the Hessian $\H(\theta)$ as
\begin{equation*}
    \H(\theta) = -\bigg(\frac{1}{n}\sum_{v=1}^n (1-\gamma)\varphi(\mathbf X_v)\varphi(\mathbf X_v)^\tr + \frac{1}{2}\frac{\gamma}{b-a}\,G_0 + 2\lambda_n\,G_q\bigg).
\end{equation*}
Since $\varphi(\mathbf X_v)\varphi(\mathbf X_v)^\tr$ is positive semi-definite for all $v\in{1,\ldots,n}$ and $G_0$ and $G_q$ are positive semi-definite, $\H(\theta)$ is negative semi-definite for all $\theta\in\R^{m+k+1}$.
\subsection{Implementation of Generalized Score Matching}
The implementation proceeds very similarly to the Simple Score Matching method. We recall that for the Generalized Score Matching estimator, we have
\begin{gather*}
     \ell(\psi; \mathbf X_1 , \ldots, \mathbf X_n) \\
     = \frac{1}{n}\sum_{v=1}^n \bigg(\!- \kappa\big(\phi(\mathbf X_v)\big)\psi'(\mathbf X_v) - \phi'(\mathbf X_v)\kappa'\big(\phi(\mathbf X_v)\big)\psi(\mathbf X_v) - \frac{1}{2}\kappa\big(\phi(\mathbf X_v)\big)\psi(\mathbf X_v)^2\bigg)
\end{gather*}
so that
\begin{gather*}
    \mathcal L(\theta) = \frac{1}{n}\sum_{v=1}^n \bigg(\!- \kappa\big(\phi(\mathbf X_v)\big)\theta^\tr \varphi'(\mathbf X_v) - \phi'(\mathbf X_v)\kappa'\big(\phi(\mathbf X_v)\big)\theta^\tr \varphi(\mathbf X_v) \\
    - \frac{1}{2}\kappa\big(\phi(\mathbf X_v)\big)\big(\theta^\tr \varphi(\mathbf X_v)\big)^2\bigg) - \lambda_n\,\theta^\tr G_q \theta.
\end{gather*}

For fixed $\theta$, we calculate the gradient and the Hessian of $\mathcal L$ at $\theta$. We have
\begin{gather*}
    \big(\S(\theta)\big)_i = \frac{\partial}{\partial\theta_i} \mathcal L(\theta) = \frac{1}{n}\sum_{v=1}^n \bigg(\!- \kappa\big(\phi(\mathbf X_v)\big)\varphi_i'(\mathbf X_v) - \phi'(\mathbf X_v)\kappa'\big(\phi(\mathbf X_v)\big)\varphi_i(\mathbf X_v) \\
    - \kappa\big(\phi(\mathbf X_v)\big)\varphi_i(\mathbf X_v)\big(\theta^\tr \varphi(\mathbf X_v)\big)\bigg) - 2\lambda_n\big(G_q \theta\big)_i
\end{gather*}
for $i=0,\ldots,m+k$, and
\begin{gather*}
    \big(\H(\theta)\big)_{i,j} = \frac{\partial^2}{\partial\theta_i\partial\theta_j} \mathcal L(\theta) = \frac{1}{n}\sum_{v=1}^n \Big(\!- \kappa\big(\phi(\mathbf X_v)\big)\varphi_i(\mathbf X_v)\varphi_j(\mathbf X_v)\Big) - 2\lambda_n\big(G_q\big)_{i,j},
\end{gather*}
for $i=0,\ldots,m+k$ and $j=0,\ldots,m+k$.

Now, we verify that $\mathcal L$ verifies the hypotheses of \autoref{prop:newton_global_max}. First, we showed that $\mathcal L$ is twice differentiable at every $\theta\in\R^{m+k+1}$.

Let $\theta\in\R^{m+k+1}$. We rewrite the Hessian $\H(\theta)$ as
\begin{equation*}
    \H(\theta) = -\bigg(\frac{1}{n}\sum_{v=1}^n \kappa\big(\phi(\mathbf X_v)\big)\varphi(\mathbf X_v)\varphi(\mathbf X_v)^\tr + 2\lambda_n\,G_q\bigg).
\end{equation*}
Since $\varphi(\mathbf X_v)\varphi(\mathbf X_v)^\tr$ is positive semi-definite for all $v\in{1,\ldots,n}$ and $G_q$ is positive semi-definite, $\H(\theta)$ is negative semi-definite for all $\theta\in\R^{m+k+1}$.
\subsection{Hyperparameter selection}
The algorithms presented rely heavily on the choice of the regularization parameter $\lambda_n$, which determines the strength of the regularization induced by the penalty term $J_q(h)$.

While we have requirements on the asymptotic behavior of $\lambda_n$ as $n\rightarrow\infty$, for a fixed sample size $n$ the value of $\lambda_n$ must be chosen from the observed data. In this section, we present two classical methods to select $\lambda_n$: simple validation and $K$-fold cross-validation. In both cases, we compare candidate values of $\lambda_n$ by evaluating how well the estimated density performs on unseen data. Since all the methods ultimately produce density estimates, we use the average log-likelihood on the validation sets to compare the different values of $\lambda_n$.

Unlike the standard simple validation and $K$-fold cross-validation procedures, we choose to resample the splits of the original dataset for each candidate value of $\lambda_n$. We do this in order to make the algorithms more robust against potentially "unlucky" splits.

In general, we expect $\lambda_n$ to decrease as $n$ increases. Since both procedures select $\lambda_n$ by fitting the estimator on training samples of size less than $n$, we expect the selected value to be larger than the one that would be optimal when using all $n$ observations. Nevertheless, we use the selected $\lambda_n$ unchanged on the full dataset. Using a slightly stronger regularization reduces the risk of overfitting and leads to more stable estimates and it is thus a more conservative choice.

We assume to have available $n$ iid realizations of $\mathbf X$, denoted as $\mathbf X_1,\dots,\mathbf X_n$.

\paragraph{Simple validation.}
For a given method for probability density estimation, we consider a grid of candidates for $\lambda_n$. For each candidate value $\lambda_n$, we split the original sample randomly into a training set $\mathcal D_{\mathrm{tr}}$ and a validation set $\mathcal D_{\mathrm{val}}$. We use a fixed split ratio, for example $80\%/20\%$. We fit the estimator on the training set and then evaluate its performance on the validation set through the average log-likelihood
\begin{equation*}
    U_{\mathrm{val}}(\lambda_n)
    = \frac{1}{|\mathcal D_{\mathrm{val}}|}\sum_{\mathbf X^{\mathrm{val}}\in\mathcal D_{\mathrm{val}}}\log \hat f_{\lambda_n,\mathrm{tr}}(\mathbf X^{\mathrm{val}}),
\end{equation*}
where $\hat f_{\lambda_n,\mathrm{tr}}$ denotes the density estimate obtained using the considered method on the training sample using penalty parameter $\lambda_n$. We then select the value $\hat\lambda_n$ that maximizes $U_{\mathrm{val}}(\lambda_n)$ over the grid. After selecting $\hat\lambda_n$, we refit the estimator on the full sample using $\hat\lambda_n$, and we take the resulting density as the final estimate.

We summarize the procedure in \autoref{alg:simple_validation}.

\begin{algorithm}[H]
\caption{Simple validation for selecting $\lambda_n$}
\label{alg:simple_validation}
\begin{algorithmic}[1]
\REQUIRE Method for density estimation, sample $\mathbf X_1,\ldots,\mathbf X_n$, grid of candidates $\Lambda =\{\lambda_n^{(1)},\ldots,\lambda_n^{(M)}\}$, split ratio $\rho\in(0,1)$.
\FOR{$j=1,\ldots,M$}
    \STATE Randomly split the sample into a training set $\mathcal D_{\mathrm{tr}}^{(j)}$ of size $\lfloor \rho n\rfloor$ and a validation set $\mathcal D_{\mathrm{val}}^{(j)}$ of size $n-\lfloor \rho n\rfloor$.
    \STATE Use the method for density estimation on $\mathcal D_{\mathrm{tr}}^{(j)}$ using penalty $\lambda_n^{(j)}$ and obtain the density estimate $\hat f_{\lambda_n^{(j)},\mathrm{tr}}$.
    \STATE Compute the validation utility
    \begin{equation*}
        U_{\mathrm{val}}(\lambda_n^{(j)})
        = \frac{1}{|\mathcal D_{\mathrm{val}}^{(j)}|}\sum_{\mathbf X^{\mathrm{val}}\in\mathcal D_{\mathrm{val}}^{(j)}}\log \hat f_{\lambda_n^{(j)},\mathrm{tr}}(\mathbf X^{\mathrm{val}}).
    \end{equation*}
\ENDFOR
\STATE Set $\displaystyle \hat\lambda_n \gets \argmax_{\lambda_n\in\Lambda} U_{\mathrm{val}}(\lambda_n)$.
\RETURN $\hat\lambda_n$.
\end{algorithmic}
\end{algorithm}

\paragraph{$K$-fold cross-validation.}
Simple validation depends on a single split between training and validation data, and the resulting choice of $\lambda_n$ can be noisy, especially if $n$ is small. A way to reduce the variability is to use $K$-fold cross-validation.

For a given method for probability density estimation, we consider a grid of candidates for $\lambda_n$. We fix an integer $K\geq 2$. For each candidate value $\lambda_n$, we partition the original sample randomly into $K$ disjoint subsets (folds) $\mathcal D_1,\ldots,\mathcal D_K$ of approximately equal size. For each fold $r\in\{1,\ldots,K\}$, we fit the estimator on the training set obtained by removing the $r$-th fold,
\begin{equation*}
    \mathcal D_{\mathrm{tr}}^{(r)}=\bigcup_{\substack{s=1\\ s\neq r}}^{K}\mathcal D_s,
\end{equation*}
and then evaluate its performance on the validation fold $\mathcal D_{\mathrm{val}}^{(r)}=\mathcal D_r$ through the average log-likelihood
\begin{equation*}
    U_{\mathrm{val}}^{(r)}(\lambda_n)
    = \frac{1}{|\mathcal D_{\mathrm{val}}^{(r)}|}\sum_{\mathbf X^{\mathrm{val}}\in\mathcal D_{\mathrm{val}}^{(r)}}\log \hat f_{\lambda_n,\mathrm{tr}}^{(r)}(\mathbf X^{\mathrm{val}}),
\end{equation*}
where $\hat f_{\lambda_n,\mathrm{tr}}^{(r)}$ denotes the density estimate obtained using the considered method on $\mathcal D_{\mathrm{tr}}^{(r)}$ using penalty parameter $\lambda_n$. We then average the utilities over the $K$ folds and define
\begin{equation*}
    U_{\mathrm{CV}}(\lambda_n) = \frac{1}{K}\sum_{r=1}^{K} U_{\mathrm{val}}^{(r)}(\lambda_n).
\end{equation*}
We select the value $\hat\lambda_n$ that maximizes $U_{\mathrm{CV}}(\lambda_n)$ over the grid. After selecting $\hat\lambda_n$, we refit the estimator on the full sample using $\hat\lambda_n$, and we take the resulting density as the final estimate.
 
We summarize the procedure in \autoref{alg:kfold_cv}.

\begin{algorithm}[H]
\caption{$K$-fold cross-validation for selecting $\lambda_n$}
\label{alg:kfold_cv}
\begin{algorithmic}[1]
\REQUIRE Method for density estimation, sample $\mathbf X_1,\ldots,\mathbf X_n$, grid of candidates $\Lambda =\{\lambda_n^{(1)},\ldots,\lambda_n^{(M)}\}$, number of folds $K$.
\FOR{$j=1,\ldots,M$}
    \STATE Randomly partition the sample into $K$ folds $\mathcal D^{(j)}_1,\ldots,\mathcal D^{(j)}_K$ of approximately equal size. \\[-1\baselineskip]
    \STATE Set $U_{\mathrm{CV}}(\lambda_n^{(j)}) \gets 0$.
    \FOR{$r=1,\ldots,K$}
        \STATE Set $\mathcal D_{\mathrm{tr}}^{(j,r)} \gets \bigcup_{s\neq r}\mathcal D^{(j)}_s$ and $\mathcal D_{\mathrm{val}}^{(j,r)}\gets \mathcal D^{(j)}_r$.
        \STATE Use the method for density estimation on $\mathcal D_{\mathrm{tr}}^{(j,r)}$ using penalty $\lambda_n^{(j)}$ and obtain the density estimate $\hat f_{\lambda_n^{(j)},\mathrm{tr}}^{(r)}$.
        \STATE Compute
        \begin{equation*}
            U_{\mathrm{val}}^{(r)}(\lambda_n^{(j)})
            = \frac{1}{|\mathcal D_{\mathrm{val}}^{(j,r)}|}\sum_{\mathbf X^{\mathrm{val}}\in\mathcal D_{\mathrm{val}}^{(j,r)}}\log \hat f_{\lambda_n^{(j)},\mathrm{tr}}^{(r)}(\mathbf X^{\mathrm{val}}).
        \end{equation*}
        \STATE Update $U_{\mathrm{CV}}(\lambda_n^{(j)}) \gets U_{\mathrm{CV}}(\lambda_n^{(j)}) + \frac{1}{K}U_{\mathrm{val}}^{(r)}(\lambda_n^{(j)})$.
    \ENDFOR
\ENDFOR
\STATE Set $\displaystyle \hat\lambda_n \gets \argmax_{\lambda_n\in\Lambda} U_{\mathrm{CV}}(\lambda_n)$.
\RETURN $\hat\lambda_n$.
\end{algorithmic}
\end{algorithm}
\subsection{Complexity analysis}\label{sec:complexity}
In this section, we analyze the time complexity of the algorithms. For simplicity, we treat $m$, $\alpha$, and $q$ as constants. In addition, since the Newton-Raphson method can take a variable number of iterations to converge, we study the cost of a single iteration.

We discuss only the Simple Maximum Likelihood and the Simple Score Matching methods. The shifted versions have the same time complexity as their simple counterparts, and Generalized Score Matching has the same time complexity as Simple Score Matching.

The penalty matrix $G_q$ is computed only once before the first iteration. The construction of the matrix $G_q$ is executed as explained in \autoref{sec:penalty} and has a cost of $O(m^2k^2) = O(k^2)$, since $m$ is constant.

In addition, we recall that evaluating $\theta^\tr\varphi(x)$ requires computing all basis functions at $x$, and therefore it has time complexity $O(m+k)=O(k)$, since $m$ is constant.

\paragraph{Simple Maximum Likelihood.} We first consider the cost of Gauss-Legendre quadrature. Using an $\alpha$-point Gauss-Legendre quadrature rule on an interval has a cost of $O(\alpha^2+\alpha\,\mathrm{cost}(f))$, where $O(\alpha^2)$ is the cost of computing the roots of the Legendre polynomial of order $\alpha$ and $\mathrm{cost}(f)$ is the cost of evaluating the integrand $f$.

In order to calculate the entries of $A$, we use the Gauss-Legendre quadrature on integrands that contain the term $\exp(\theta^\tr\varphi(x))$. Then, since $A$ has $O(mk)$ entries, the cost of calculating $A$ is $O(mk(\alpha^2+\alpha k)) = O(k^2)$, since $m$ and $\alpha$ are constants.

Once $A$ is available, computing $I^{(0)}$ has cost $O(k)$, computing $I^{(1)}$ has cost $O(k^2)$, and computing $I^{(2)}$ has cost $O(k^3)$.

Evaluating the objective function $\mathcal L(\theta)$ has a cost of $O(nk+ k^2)$ as it requires computing $\frac{1}{n}\sum_{v=1}^n \theta^\tr\varphi(\mathbf X_v)$ and evaluating the quadratic penalty $\theta^\tr G_q\theta$. Computing the gradient has the same complexity $O(nk+k^2)$, while assembling the Hessian costs $O(k^2)$.

The final step of the iteration is solving the linear system to find the update direction, which costs $O(k^3)$. Therefore, the overall cost per iteration for Simple Maximum Likelihood is
\begin{equation*}
    O\big(k^3 + nk\big).
\end{equation*}
In particular, for fixed $k$ the complexity grows linearly in $n$, whereas for fixed $n$ the dependence on $k$ is dominated by the cubic cost of computing $I^{(2)}$ and solving the linear system.

\paragraph{Simple Score Matching.}
First, we consider the cost of evaluating $\mathcal L(\theta)$. This requires computing $\frac{1}{n}\sum_{v=1}^n \theta^\tr\varphi'(\mathbf X_v)$, $\frac{1}{n}\sum_{v=1}^n \big(\theta^\tr\varphi(\mathbf X_v)\big)^2$ and the quadratic penalty $\theta^\tr G_q\theta$, that, respectively, have cost $O(kn)$, $O(kn)$, and $O(k^2)$. Computing the gradient has the same complexity. In order to compute the Hessian, we have to evaluate
\begin{equation*}
    \frac{1}{n}\sum_{v=1}^n \varphi(\mathbf X_v)\varphi(\mathbf X_v)^\tr,
\end{equation*}
which is a sum of $n$ rank-one matrices of size $k\times k$, and therefore it has cost $O(nk^2)$.

The final step of the iteration is solving the linear system to find the update direction, which costs $O(k^3)$. Therefore, the overall cost per iteration for Simple Score Matching is
\begin{equation*}
    O\big(k^3 + nk^2\big).
\end{equation*}
In particular, for fixed $k$ the complexity grows linearly in $n$, whereas for fixed $n$ the dependence on $k$ is dominated by the cubic cost of solving the linear system.

Even though this bound is worse than the corresponding bound for Simple Maximum Likelihood, we expect Simple Score Matching to run faster in practice for small $k$, since the iterations require less steps and do not require numerical integration.

A possible way to reduce the time complexity of both methods is to replace Newton-Raphson with a first-order optimizer such as gradient descent or Adam optimizer. This would avoid computing the Hessian and solving a linear system, at the price of potentially requiring more iterations to converge.