\subsection{Penalty term}\label{sec:penalty}
In this section, we show how we can compute the penalty term of the penalized utility functions, as it is the same in every method. We recall that
\begin{equation*}
    J_q(g) = \int_a^b \big(g^{(q)}(x)\big)^2 \de x = \|g^{(q)}\|_2^2
\end{equation*}

Let $\theta$ be the coefficient vector of $g$. Then we have
\begin{equation*}
    J_q(g) = \int_a^b \Big(\sum_{i=0}^{m+k} \theta_i\varphi_i^{(q)}(x)\Big)^2 \de x = \sum_{i=0}^{m+k}\sum_{j=0}^{m+k} \theta_i \theta_j \int_a^b\varphi_i^{(q)}(x)\varphi_j^{(q)}(x) \de x = \theta^\tr G_q \theta
\end{equation*}
where $G_q\in\R^{(k+m+1)\times(k+m+1)}$ is the symmetric positive semi-definite matrix with entries
\begin{equation}
    (G_q)_{i,j} = \int_a^b \varphi_i^{(q)}(x)\varphi_j^{(q)}(x)\,\de x,
    \qquad\substack{\displaystyle i=0,\ldots,m+k,\\ \displaystyle j=0,\ldots,m+k.}
\end{equation}

In the truncated power basis, $q$ derivatives are identically zero. More precisely, for $0\leq i<q$,
\begin{equation*}
    \varphi_i^{(q)}\equiv0
\end{equation*}
For $q\leq i \leq m$,
\begin{equation*}
    \varphi_i^{(q)}(x) = \frac{i!}{(i-q)!}\,x^{i-q}.
\end{equation*}
For $j=1,\ldots,k$,
\begin{equation*}
    \varphi_{m+j}^{(q)}(x)
    = \frac{m!}{(m-q)!}(x-t_j)_+^{m-q},
\end{equation*}
with the convention that $(x-t_j)_+^{0}=\I\{x>t_j\}$.

It follows that the first $q$ rows and columns of $G_q$ are filled with $0$. In addition, every entry of $G_q$ can be calculated exactly by integrating polynomials. For example, take $i\in\{q,\ldots,m\}$ and $j\in\{1,\ldots,k\}$, then
\begin{gather*}
    (G_q)_{i,m+j} = \frac{i!}{(i-q)!}\frac{m!}{(m-q)!} \int_a^b x^{i-q} (x-t_j)_+^{m-q} \de x \\
    = \frac{i!}{(i-q)!}\frac{m!}{(m-q)!} \int_{t_j}^b x^{i-q} (x-t_j)^{m-q} \de x \\
    = \frac{i!}{(i-q)!}\frac{m!}{(m-q)!} \int_{t_j}^b \sum_{v=0}^{m-q} \tbinom{m-q}{v} x^{i+m-2q-v} (-t_j)^v \de x \\
    = \frac{i!}{(i-q)!}\frac{m!}{(m-q)!}\sum_{v=0}^{m-q} \tbinom{m-q}{v}(-t_j)^v\int_{t_j}^b x^{i+m-2q-v} \de x \\
    = \frac{i!}{(i-q)!}\frac{m!}{(m-q)!}\sum_{v=0}^{m-q} \tbinom{m-q}{v}(-t_j)^v \frac{b^{i+m-2q-v+1} - t_j^{i+m-2q-v+1}}{i+m-2q-v+1}.
\end{gather*}

With a bit more work one can calculate also $(G_q)_{i,j}$ for $i,j\in\{q,\ldots,m\}$ and $(G_q)_{m+i,m+j}$ for $i,j\in\{1,\ldots,k\}$. 

It's important to notice that $G_q$ depends only on the basis and does not change between iterations. Therefore, it can be computed once and stored, so that the penalty evaluation reduces to evaluating the quadratic form $\theta^\tr G_q \theta$.

The matrix $G_q$ can be used to evaluate all integrals of the form
\begin{equation*}
    \int_a^b g^{(q)}(x)h^{(q)}(x)\de x.
\end{equation*}
Let $\theta$ and $\theta^*$ be the coefficient vectors of $g$ and $h$, respectively. Then
\begin{equation*}
    \int_a^b g^{(q)}(x)h^{(q)}(x)\de x = \theta^\tr G_q \theta^*.
\end{equation*}

In particular, we have that
\begin{equation}\label{eq:normal_integral}
    \int_a^b g(x)\de x = \int_a^b 1\cdot g(x)\de x = e_0^\tr G_0 \theta,
\end{equation}
where $e_0=(1,0,\ldots,0)^\tr\in\R^{m+k+1}$ is the first canonical basis vector, and
\begin{equation}\label{eq:squared_integral}
    \int_a^b g(x)^2\de x = \theta^\tr G_0 \theta,
\end{equation}