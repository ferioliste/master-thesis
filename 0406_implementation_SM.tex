\subsection{Implementation of Simple Score Matching}
We recall that for the Simple Score Matching estimator, we have
\begin{equation*}
     \ell(\psi; \mathbf X_1 , \ldots, \mathbf X_n) = \frac{1}{n}\sum_{v=1}^n \bigg(\!- \psi'(\mathbf X_v) - \frac{1}{2}\psi(\mathbf X_v)^2\bigg),
\end{equation*}
so that
\begin{equation*}
    \mathcal L(\theta) = \frac{1}{n}\sum_{v=1}^n \bigg(\!- \theta^\tr \varphi'(\mathbf X_v) - \frac{1}{2}\big(\theta^\tr \varphi(\mathbf X_v)\big)^2\bigg) - \lambda_n\,\theta^\tr G_q \theta,
\end{equation*}
where $\varphi'(x)\in\R^{m+k+1}$ is such that $\big(\varphi'(x)\big)_i = \varphi_i'(x)$ for $i=0,\ldots,m+k$.

We recall that $\varphi_0' \equiv 0$. For $i=1,\ldots,m$,
\begin{equation*}
    \varphi_i'(x) = i\,x^{i-1}.
\end{equation*}
For $j=1,\ldots,k$,
\begin{equation*}
    \varphi_{m+j}'(x)
    = m(x-t_j)_+^{m-1}.
\end{equation*}

For fixed $\theta$, we calculate the gradient and the Hessian of $\mathcal L$ at $\theta$. We have
\begin{equation*}
    \big(\S(\theta)\big)_i = \frac{\partial}{\partial\theta_i} \mathcal L(\theta) = \frac{1}{n}\sum_{v=1}^n \bigg(\!- \varphi_i'(\mathbf X_v) - \varphi_i(\mathbf X_v)\big(\theta^\tr \varphi(\mathbf X_v)\big)\bigg) - 2\lambda_n\big(G_q \theta\big)_i
\end{equation*}
for $i=0,\ldots,m+k$, and
\begin{gather*}
    \big(\H(\theta)\big)_{i,j} = \frac{\partial^2}{\partial\theta_i\partial\theta_j} \mathcal L(\theta) = \frac{1}{n}\sum_{v=1}^n \big(\!- \varphi_i(\mathbf X_v)\varphi_j(\mathbf X_v)\big) - 2\lambda_n\big(G_q\big)_{i,j},
\end{gather*}
for $i=0,\ldots,m+k$ and $j=0,\ldots,m+k$.

Now, we verify that $\mathcal L$ verifies the hypotheses of \autoref{prop:newton_global_max}. First, we showed that $\mathcal L$ is twice differentiable at every $\theta\in\R^{m+k+1}$.

Let $\theta\in\R^{m+k+1}$. We rewrite the Hessian $\H(\theta)$ as
\begin{equation*}
    \H(\theta) = -\bigg(\frac{1}{n}\sum_{v=1}^n \varphi(\mathbf X_v)\varphi(\mathbf X_v)^\tr + 2\lambda_n\,G_q\bigg).
\end{equation*}
Since $\varphi(\mathbf X_v)\varphi(\mathbf X_v)^\tr$ is positive semi-definite for all $v\in{1,\ldots,n}$ and $G_q$ is positive semi-definite, $\H(\theta)$ is negative semi-definite for all $\theta\in\R^{m+k+1}$.

We summarize the complete procedure in \autoref{alg:simple_SM_newton}.

\begin{algorithm}[H]
\caption{Algorithm for Simple Score Matching}
\label{alg:simple_SM_newton}
\begin{algorithmic}[1]
\REQUIRE Data $\mathbf X_1,\ldots,\mathbf X_n$, initial value $\theta^{(0)}\in\R^{m+k+1}$, step size $\zeta\in(0,1]$, tolerance $\varepsilon>0$, maximum number of iterations $I_{\max}$.
\STATE Set $i\gets 0$.
\STATE Set $L_{\text{new}}\gets -\infty$.
\WHILE{$i<I_{\max}$}
    \STATE Set $L_{\text{old}} \gets L_{\text{new}}$.
    \STATE Compute $\mathcal L(\theta^{(i)})$.
    \STATE Set $L_{\text{new}} \gets \mathcal L(\theta^{(i)})$.
    \IF{i>0}
        \STATE Set $\displaystyle r \gets \frac{L_{\text{new}} - L_{\text{old}}}{\max\{|L_{\text{old}}|,\,1\}}$
        \IF{$r<\varepsilon$}
            \STATE \textbf{break}
        \ENDIF
    \ENDIF
    \STATE Compute the gradient $\S(\theta^{(i)})$.
    \STATE Compute the Hessian $\H(\theta^{(i)})$.
    \STATE Solve the linear system $\H(\theta^{(i)})\,d^{(i)} = \S(\theta^{(i)})$ for $d^{(i)}\in\R^{m+k+1}$.
    \STATE Set $\theta^{(i+1)} \gets \theta^{(i)} + \zeta\, d^{(i)}$.
    \STATE Set $i\gets i+1$.
\ENDWHILE
\RETURN $\theta^{(i)}$
\end{algorithmic}
\end{algorithm}