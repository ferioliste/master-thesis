\section{Numerical Experiments}
In this last chapter, we present the results of the numerical experiments we conducted to compare the five methods for probability density estimation.

We first study the methods on synthetic data, where the ground-truth density is known, and then we consider two real-world applications.

We introduce two unnormalized densities $f_1$ and $f_2$, defined on $[a,b]=[-2,2]$ by
\begin{equation*}
    f_1(x) = e^{-(x-2)^4} + \tfrac{1}{2}e^{-(x+2)^4} + 0.05
\end{equation*}
and
\begin{equation*}
    f_2(x) =
    \Big(
        e^{-\sqrt{16(x - 0.5)^2 + 1}}
        + 2\,e^{-\sqrt{16(x + 0.5)^2 + 1}}
        + 4\,e^{-\sqrt{128(x - 1.25)^2 + 1}}
    \Big)(4 - x^2).
\end{equation*}
We choose $f_1$ and $f_2$ on purpose to be non-polynomial $C^\infty$ functions. This is important since otherwise estimating them would be a trivial task and a not so useful benchmark for the considered methods.

\begin{figure}[hbt]
\centering
\includegraphics[width=.8\linewidth]{plots/distributions.pdf}
\caption{Normalized target densities $f_1$ and $f_2$ on $[a,b]=[-2,2]$, used in the synthetic experiments.}
\label{fig:distributions}
\end{figure}

We have that $f_1$ is bounded away from $0$ and infinity, while $f_2$ satisfies the boundary condition $f_2(a)=f_2(b)=0$. These properties determine which estimation algorithms can be applied. In particular, $f_1$ can be used with the Simple Maximum Likelihood, Shifted Maximum Likelihood, and Generalized Score Matching methods, while $f_2$ can be used with the Shifted Maximum Likelihood, Simple Score Matching, Shifted Score Matching, and Generalized Score Matching methods.

In order to generate samples from $f_1$ and $f_2$, we use the acceptance-rejection method, which only requires knowledge of the unnormalized density. This allows us to compare the estimated densities with the normalized ground-truth.

Throughout this chapter, we fix the spline degree to $m=3$, as in \cite{kooperberg1992}, and we use the penalty order $q=2$. In our experiments, provided that $1/2<q\leq m$, we did not observe substantial differences in the qualitative behavior of the estimators when varying $(m,q)$. For this reason, we keep these choices fixed in all the numerical results reported below.

In all experiments we use uniformly spaced knots on $[a,b]$. For a given number of interior knots $k$, we take $t_i = a+i(b-a)/(k+1)$ for $i=0,\ldots,k+1$. 

In addition, we use the following fixed parameters:
\begin{itemize}
    \item We initialize the optimizations at $\theta^{(0)}=0$.
    \item We use Gauss-Legendre quadrature with $\alpha=6$ nodes.
    \item In the Newton-Raphson method use a fixed step size $\zeta=0.1$, we perform at most $I_{\max}=1000$ iterations, and we stop the iterations when $r$ falls below $10^{-6}$
    \item For Generalized Score Matching, we set $\beta=2$, as in \cite{gen_sm}.
\end{itemize}

\subsection{Choice of the number of knots}
The first question we address is how the number of interior knots $k$ should be related to the sample size $n$ in practice. To keep the comparison simple, in this subsection we focus on two representative methods, the Simple Maximum Likelihood and the Simple Score Matching estimators. Since the applicability of these methods differ, we use the first method on samples generated from $f_1$ and the second on samples generated from $f_2$.

We consider three values of the number of interior knots, $k\in\{5,10,20\}$, and three sample sizes, $n\in\{100,1000,10000\}$, for a total of nine possible configurations.

For each pair $(n,k)$, we perform $5$ pilot runs in order to select the penalty parameter $\lambda_n$ by k-fold cross-validation. More precisely, at each pilot run we
\begin{enumerate}
    \item generate a new sample of size $n$ from the ground-truth distribution,
    \item select $\lambda_n$ by k-fold cross-validation,
    \item store the selected value of $\lambda_n$.
\end{enumerate}
At the end of the $5$ pilot runs, we set $\lambda_n$ equal to the largest of the selected values, as we prefer a conservative penalty.

After fixing $\lambda_n$ for each configuration $(n,k)$, we generate three additional independent samples and fit the estimator on each of them. The resulting estimated densities are then plotted and compared among all combinations of $n$ and $k$. The outcomes for the Simple Maximum Likelihood estimator and the Simple Score Matching estimator are shown in \autoref{fig:n_vs_k_MLE} and \autoref{fig:n_vs_k_SM}, respectively.

From these experiments, we make the following observations. As expected, the quality of the estimates improves as the sample size $n$ increases. We see that the variability across the three fitted curves decreases and the estimated densities become closer to the ground-truth. Moreover, increasing the number of interior knots $k$ generally improves the approximation by enlarging the spline space, but only up to a point. In particular, if $k$ is too small the estimator may not be flexible enough to capture local features of the target distribution. This is clearly visible for Simple Score Matching (\autoref{fig:n_vs_k_SM}), where for $k=5$ the fitted density fails to capture the smallest mode of the ground-truth. For this reason, in all subsequent experiments we fix $k=10$. More generally, we expect that relatively small values of $k$ are sufficient for smooth and simple targets, while larger values are needed when the distribution presents sharper features, multiple modes, or other local irregularities.

\begin{figure}[H]
\centering
\includegraphics[width=.89\linewidth]{plots/n_vs_k_MLE.pdf}
\caption{Effect of the sample size $n$ and the number of interior knots $k$ on the Simple Maximum Likelihood estimator. The ground-truth density (black) is compared to three estimated densities (colored), obtained from three independent samples.}
\label{fig:n_vs_k_MLE}
\end{figure}

\begin{figure}[H]
\centering
\includegraphics[width=.89\linewidth]{plots/n_vs_k_SM.pdf}
\caption{Effect of the sample size $n$ and the number of interior knots $k$ on the Simple Score Matching estimator. The ground-truth density (black) is compared to three estimated densities (colored), obtained from three independent samples.}
\label{fig:n_vs_k_SM}
\end{figure}

\newpage
\subsection{Algorithm comparison}
In this section, we compare the performance of all five density estimation methods presented. We fix the number of interior knots to $k=10$, as motivated in the previous section, and we consider three sample sizes, $n\in\{100,1000,10000\}$.

Each method is tested on every target density for which its assumptions are satisfied. This yields a total of seven configurations, since two of the methods can be applied to both $f_1$ and $f_2$. For the shifted methods we use $\gamma = 0.05$.

For each configuration and each sample size $n$, we first perform $5$ pilot runs in order to select the penalty parameter $\lambda_n$ by $5$-fold cross-validation, as done in the previous section. After fixing $\lambda_n$, we generate three additional independent samples and fit the estimator on each of them.

The outcomes for the targets $f_1$ and $f_2$ are reported in \autoref{fig:alg_vs_n_f1} and \autoref{fig:alg_vs_n_f2}, respectively.

From these experiments, we make a few considerations. When the target distribution is $f_1$, all methods provide a good approximation of the ground-truth across the three sample sizes considered. The only mehtod that displays a slightly worse behavior is Generalized Score Matching, which seems to behave worse near the boundary of the support. In general, we \linebreak
\begin{figure}[H]
\centering
\includegraphics[width=.9\linewidth]{plots/alg_vs_n_MLE.pdf}
\caption{Comparison of three methods on the target density $f_1$ for $k=10$ and $n\in\{100,1000,10000\}$. The ground-truth density (black) is compared to three estimated densities (colored), obtained from three independent samples.}
\label{fig:alg_vs_n_f1}
\end{figure}
\begin{figure}[H]
\centering
\includegraphics[width=.9\linewidth]{plots/alg_vs_n_SM.pdf}
\caption{Comparison of four methods on the target density $f_2$ for $k=10$ and $n\in\{100,1000,10000\}$. The ground-truth density (black) is compared to three estimated densities (colored), obtained from three independent samples.}
\label{fig:alg_vs_n_f2}
\end{figure}
notice an improvement when increasing $n$ from $100$ to $1000$, but the gain obtained by further increasing the sample size from $1000$ to $10000$ is smaller.

When the target distribution is $f_2$, all methods again provide a good approximation of the ground-truth. However, the score matching based estimators seem to converge to a distribution that is slightly shifted with respect to the ground truth, although this effect becomes less pronounced as $n$ increases. Somewhat surprisingly, the Shifted Maximum Likelihood estimator exhibits the best qualitative behavior, despite the fact of being biased. Finally, again, the improvement between $n=1000$ and $n=10000$ is minor.

In order to compare the quality of the estimations quantitatively, for each configuration and each sample size $n$ we performed an additional $15$ independent tests. For each one, we computed the $L_2$ error between the estimated density and the ground-truth density, and we summarize the results using boxplots of the resulting errors in \autoref{fig:boxplot}. Overall, the error decreases as $n$ increases, as expected. The only exception is the Simple Score Matching estimator, for which the median error slightly increases when passing from $n=1000$ to $n=10000$. Finally, as $n$ grows, the variability of the error also decreases, which we had already noticed in the qualitative plots. \vspace{\baselineskip}

\begin{figure}[hbt]
\centering
\includegraphics[width=.99\linewidth]{plots/boxplots.pdf}
\caption{Boxplots of the $L_2$ error for the seven method-target configurations and $n\in\{100,1000,10000\}$. Each boxplot is computed from $15$ independent repetitions.}
\label{fig:boxplot}
\end{figure}
\subsection{The impact of the penalty parameter}
In the previous two subsections, we selected $\lambda_n$ as robustly as possible in order to compare the different methods under ideal conditions. In this subsection, we study instead the effect of the penalty parameter on the fitted density. As before, for simplicity, we focus only on the Simple Maximum Likelihood estimator and the Simple Score Matching estimator.

In order to highlight more clearly the impact of $\lambda_n$, we fix a small sample size, $n=100$, where the bias-variance trade-off induced by the penalty is most visible. We fit the Simple Maximum Likelihood estimator on samples generated from $f_1$, and the Simple Score Matching estimator on samples generated from $f_2$. For each method, we consider the values
\begin{equation*}
    \lambda_n \in \{10^{-6},10^{-5},10^{-4},10^{-3},10^{-2},10^{-1}\}.
\end{equation*}
The results are shown in \autoref{fig:lambda_MLE} and \autoref{fig:lambda_SM}, where the estimated densities are compared to the empirical histogram of the sample (30 bins), in order to highlight under- and over-regularization.

\begin{figure}[H]
\centering
\includegraphics[width=.9\linewidth]{plots/lambda_MLE.pdf}
\caption{Effect of the penalty parameter $\lambda_n$ on the Simple Maximum Likelihood estimator for $n=100$. For $\lambda_n\in\{10^{-6},10^{-5},10^{-4},10^{-3},10^{-2},10^{-1}\}$, the fitted density (red) is compared to the empirical histogram of the sample (30 bins). Using k-fold crossvalidation, we select $\lambda_n = 10^{-2}$.}
\label{fig:lambda_MLE}
\end{figure}
\vspace{2\baselineskip}
\begin{figure}[H]
\centering
\includegraphics[width=.9\linewidth]{plots/lambda_SM.pdf}
\caption{Effect of the penalty parameter $\lambda_n$ on the Simple Score Matching estimator for $n=100$. For $\lambda_n\in\{10^{-6},10^{-5},10^{-4},10^{-3},10^{-2},10^{-1}\}$, the fitted density (red) is compared to the empirical histogram of the sample (30 bins). Using k-fold crossvalidation, we select $\lambda_n = 10^{-3}$.}
\label{fig:lambda_SM}
\end{figure}
\newpage

As $\lambda_n$ increases, the estimated density becomes smoother, since the penalty induces additional regularity on the fitted function.

Looking at \autoref{fig:lambda_MLE}, for the Simple Maximum Likelihood estimator a good value for $\lambda_n$ seems to be $10^{-3}$, however, using k-fold cross validation we select $10^{-2}$. In this synthetic setting, one should always keep in mind that we know the exact target distribution in advance, and so choosing $\lambda_n$ "visually" might result to be "unfair compared to a real-data situation. In practice, when the true distribution is unknown, the choice of $\lambda_n$ must be based on data-driven procedures such as cross-validation.

Similarly, for the Simple Score Matching estimator, \autoref{fig:lambda_SM} suggests that a smaller penalty would be better, as we need enough flexibility in order to recover the three modes. However, using k-fold cross validation we select $10^{-3}$, that is not small enough to fit a distribution with three modes. In practice, we would not know that the distribution has three modes, so the estimated distribution with $\lambda_n = 10^{-3}$ would be perfectly reasonable.
\subsection{Real-world testcase}
In this last section, we present two real-world test cases. We use the \emph{NYC Yellow Taxi Trip} datase from Kaggle\footnote{Dataset available at \href{https://www.kaggle.com/datasets/elemento/nyc-yellow-taxi-trip-data}{\underline{https://www.kaggle.com/datasets/elemento/nyc-yellow-taxi-trip-data}}}. We consider two variables: the taxi pick-up time and the trip distance. For the pick-up times, we restrict the analysis to the daytime interval from $06{:}00$ to $22{:}00$ and we extract a random subsample of $n=1000$ observations. We make the working assumption that the observations are independent and identically distributed.

Since there are no times during the day in which the demand for taxis is exactly zero, it is reasonable to assume that the corresponding density is bounded away from $0$ and infinity on the considered interval. For this reason, we estimate the pick-up time density using the Simple Maximum Likelihood estimator. We select $\lambda_n=10^{-7}$ using k-fold crossvalidation. The resulting estimate is shown in \autoref{fig:arrival_time_distr}. The fitted density is trimodal: the first peak occurs around $08{:}30$, which coincides with the morning commute; the second peak appears around $14{:}00$, shortly after lunch; and the third peak is around $18{:}30$, when many people return home from work. The density increases again as $22{:}00$ approaches. A distribution of this type can be useful, for instance, to inform decisions on how many taxis should be operating at different times of the day.

We also estimate the distribution of trip distances. We remove trips with distance equal to $0$ and we discard trips longer than $13$ km, since they represent only a small fraction of the data and would dominate the scale of the plot. We select $\lambda_n=10^{-6}$ again using k-fold crossvalidation. The estimated density, shown in \autoref{fig:distance_distr}, is unimodal, with mode around $1$ km.

\begin{figure}[hbt]
\centering
\includegraphics[width=.7\linewidth]{plots/arrival_time_distr.pdf}
\caption{Estimated density of taxi pick-up times between $06{:}00$ and $22{:}00$ using the Simple Maximum Likelihood estimator on a subsample of $n=1000$ observations.}
\label{fig:arrival_time_distr}
\end{figure}

\begin{figure}[hbt]
\centering
\includegraphics[width=.7\linewidth]{plots/distance_distr.pdf}
\caption{Estimated density of taxi trip distances using the Simple Maximum Likelihood estimator on a subsample of $n=1000$ observations.}
\label{fig:distance_distr}
\end{figure}

\subsection{Runtime analysis}
In this last section, we study the runtime of the estimation algorithms and how it changes as a function of the sample size $n$ and of the number of interior knots $k$. Here, we do not analyze the Shifted Score Matching method, since its implementation has exactly the same structure as the Simple Score Matching algorithm and the observed runtimes would be essentially the same. We also do not include the KDE into the runtime study, since it is outside the scope of this thesis. In addition, KDE is not an "iteration-based" procedure, so a comparison in terms of runtime per iteration is not possible.

To analyze the dependence on $n$, we subsample from the same dataset used in the previous section and we consider
\begin{equation*}
    n\in\{100,\ 300,\ 1000,\ 3000,\ 10000,\ 30000,\ 100000\}.
\end{equation*}
For each sample size, we run the same four algorithms with $k=10$. For each combination between algorithm and sample size we perform $9$ repetitions. We randomize the order in which the configurations are executed in order to reduce dependence on external factors, such as background processes and CPU heating.

Since the Newton-Raphson method can take a variable number of iterations across configurations, a direct comparison of the total runtime would mix together computational cost and convergence speed. To separate these two effects, we record the execution times and normalize them over $1000$ iterations. Finally, we aggregate the $9$ repetitions using the median rather than the mean, since runtimes can occasionally be affected by outliers due to background processes and other sources of variability. The results are reported in \autoref{fig:runtimes} in logarithmic scale on both axes, together with the reference line $n/100$. For sufficiently large $n$, all curves become approximately parallel to this reference, suggesting that the runtime per iteration eventually grows linearly in $n$, as shown in \autoref{sec:complexity}. For approximately $n < 10^4$, the two maximum likelihood based algorithms display an almost constant runtime per iteration, which indicates that for moderate sample sizes the cost is dominated by operations that do not scale with $n$, while the linear dependence becomes visible only when $n$ is larger.

We then study the dependence on the number of interior knots $k$. For this experiments we use two independent datasets of fixed size $n=1000$, and we consider
\begin{equation*}
    k\in\{4,\ 7,\ 11,\ 18,\ 30,\ 50\}.
\end{equation*}
As before, for each combination between algorithm and number of inner knots we perform $9$ repetitions, we randomize the order in which the configurations are executed, we normalize runtimes over $1000$ iterations, and we aggregate using the median. The results are reported in \autoref{fig:runtimes} in logarithmic scale on both axes.

In this setting, the score matching based algorithms have consistently smaller runtimes for all values of $k$. 
We are not able to verify the asymptotic complexity of the algorithms when $k$ increases, since the tested values of $k$ are relatively small. Larger values of $k$ become quickly intractable on a standard computer, especially for the maximum likelihood based methods.
Overall, these results indicate that the score matching approach leads to a substantial reduction in runtime and allows one to work with richer spline spaces.

\begin{figure}[hbt]
\centering
\begin{subfigure}{.49\linewidth}
    \centering
    \includegraphics[width=\linewidth]{plots/runtimes_n.pdf}
    \label{fig:runtimes_n}
\end{subfigure}
\hfill
\begin{subfigure}{.49\linewidth}
    \centering
    \includegraphics[width=\linewidth]{plots/runtimes_k.pdf}
    \label{fig:runtimes_k}
\end{subfigure}
\caption{Median runtime per $1000$ Newton-Raphson iterations (9 repetitions per configuration). Left: varying the sample size $n$. Right: varying the number of interior knots $k$.}
\label{fig:runtimes}
\end{figure}
