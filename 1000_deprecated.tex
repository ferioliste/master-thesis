\begin{proposition}
    The functional $l$ as in \autoref{eq:ell_shifted_SM} satisfies the hypothesis of \autoref{lem:cond_2_bis}, and so condition (1) of \autoref{thm:est_error} is satisfied.
\end{proposition}

\begin{proof}\mbox{}

Similarly,
\begin{equation*}
    \mathrm{I} \leq C_2^*\|g'\|_2^2.
\end{equation*}

We want to construct a bound of the form $\|g'\|_2^2 \leq M \|g\|_2^2$ for some $M>0$. Notice that $M$ cannot depend on $n$, so in this case we cannot use the fact that in finite dimensional spaces all norms are equivalent, because the constant would depend on $\mathbb G_n$ and so on $n$.

We recall that $\mathbb G_n$ is identified by the knots vector $t\in\R^{k+2}$ and the degree $m$. We write
\begin{equation*}
    \|g'\|_2^2 = \int_a^b g'(x)^2\de x =  \sum_{i=0}^k \int_{t_i}^{t_{i+1}} g'(x)^2\de x.
\end{equation*}
We fix $i\in\{0,\dots, k\}$. Since $g\in\mathbb G$, it is a polynomial on $[t_i, t_{i+1}]$. So we have that, on $[t_i, t_{i+1}]$,
\begin{equation*}
    g(x) = \sum_{j=0}^m a_j x^j\qquad\text{and}\qquad g'(x) = \sum_{j=0}^m j a_j x^{j-1}
\end{equation*}
for some coefficients $a_0,\dots,a_m\in\R$.

Then,
\begin{equation*}
    \int_{t_i}^{t_{i+1}}g(x)^2 \de x = \sum_{j=0}^m \sum_{u=0}^m a_j a_u \int_{t_i}^{t_{i+1}} x^{j+u} \de x = \sum_{j=0}^m \sum_{u=0}^m \frac{a_j a_u}{j+u+1} (t_{i+1}^{j+u+1} - t_i^{j+u+1})
\end{equation*}
and
\begin{equation*}
    \int_{t_i}^{t_{i+1}}g'(x)^2 \de x = \sum_{j=0}^m \sum_{u=0}^m j\, u\, a_j a_u \int_{t_i}^{t_{i+1}} x^{j+u-2} \de x = \sum_{j=0}^m \sum_{u=0}^m \frac{j\, u\, a_j a_u}{j+u-1} (t_{i+1}^{j+u-1} - t_i^{j+u-1}).
\end{equation*}

We perform a term by term comparison. 






\begin{definition}[Empirical and theoretical norms]
    We define the empirical and theoretical inner products, respectively, as
    \begin{gather*}
        \langle g_1, g_2 \rangle_{\E_n} = \E_n[g_1(\mathbf X) g_2(\mathbf X)] 
        = \frac{1}{n} \sum_{i=1}^{n} g_1(\mathbf X_i) g_2(\mathbf X_i)\quad\text{and} \\
        \langle g_1, g_2 \rangle_\E = \E[g_1(\mathbf X) g_2(\mathbf X)].
    \end{gather*}
    These two inner products induce the norms
    \begin{equation*}
        \| g \|_{\E_n}^2 = \E_n[g(\mathbf X)^2] 
        = \frac{1}{n} \sum_{i=1}^{n} g(\mathbf X_i)^2\quad\text{and}\quad \| g \|_\E^2 = \E[g(\mathbf X)^2].
    \end{equation*}
\end{definition}










For $j\in\{0,\ldots,m\}$,
\begin{equation*}
    I^{(1)}_j = \int_a^b x^j\exp\!\big(\theta^\tr \varphi(x)\big)\de x = \sum_{i=0}^k \int_{t_i}^{t_{i+1}}x^j\exp\!\big(\theta^\tr \varphi(x)\big)\de x = \sum_{i=0}^k A_{i,j}.
\end{equation*}












\begin{proposition}[Sufficient condition for validity of $\ell$]
Let $\{\varphi_0,\ldots,\varphi_{m+k}\}$ be a basis of the spline space $\mathbb G$. For any spline $g\in\mathbb G$ there exists a unique coefficient vector $\theta\in\R^{m+k+1}$ such that
\begin{equation}
    g(x) = \theta^\tr \varphi(x) = \sum_{i=0}^{m+k} \theta_i\varphi_i(x),
\end{equation}
where $\varphi(x)\in\R^{m+k+1}$ is such that $\big(\varphi(x)\big)_i = \varphi_i(x)$ for $i=0,\ldots,m+k$.

We define
\begin{equation*}
    \tilde\ell(\theta;\mathbf W_1,\ldots,\mathbf W_n) = \ell\big(\theta^\tr \varphi(x);\mathbf W_1,\ldots,\mathbf W_n\big)\qquad\text{and}\qquad\tilde\Lambda(\theta) = \Lambda\big(\theta^\tr \varphi(x)\big)
\end{equation*}
for $\theta\in\R^{m+k+1}$.

Assume the following conditions hold.
\begin{enumerate}[label=\roman*)]
    \item $\displaystyle \min_{h\in W^p} \Lambda(h) = \Lambda(\eta_0)$;

    \item The Hessian $\nabla^2_{\theta} \tilde\ell(\theta;\mathbf W_1,\ldots,\mathbf W_n)$ of $\tilde\ell$ is negative semi-definite for all $\theta\in\R^{m+k+1}$ and for all possible values of $\mathbf W_1,\ldots,\mathbf W_n$;

    \item The Hessian $\nabla^2_{\theta} \tilde\Lambda(\theta)$ of $\tilde\Lambda$ is strictly negative definite for all $\theta\in\R^{m+k+1}$.
\end{enumerate}

Then $\ell$ is a valid utility function.
\end{proposition}

\begin{proof}
We verify the three properties of a valid utility function. The first property is automatically satisfied.

Let $h_1,h_2\in\mathbb G$ and $0\le\alpha\le1$. Let $\theta_1,\theta_2\in\mathbb R^{m+k+1}$ the coefficient vectors of $h_1$ and $h_2$, respectively. Since $\mathbb G$ is a linear space,
\begin{equation*}
    \alpha h_1 + (1-\alpha)h_2
= (\alpha\theta_1 + (1-\alpha)\theta_2)^\tr \varphi(x).
\end{equation*}
For the second property, we have
\begin{equation*}
    \ell(\alpha h_1 + (1-\alpha)h_2;\mathbf W_1,\ldots,\mathbf W_n)
= \tilde\ell(\alpha\theta_1 + (1-\alpha)\theta_2;
\mathbf W_1,\ldots,\mathbf W_n).
\end{equation*}
By assumption (ii), the Hessian
$\nabla^2_{\theta}\tilde\ell(\theta;\mathbf W_1,\ldots,\mathbf W_n)$
is negative semi-definite for all $\theta$ and all possible values of $\mathbf W_1,\ldots,\mathbf W_n$.
Therefore, $\tilde\ell$ is concave on $\mathbb R^{m+k+1}$, which implies
\begin{equation*}
    \tilde\ell(\alpha\theta_1 + (1-\alpha)\theta_2;\mathbf W_1,\ldots,\mathbf W_n)
    \geq
    \alpha \tilde\ell(\theta_1;\mathbf W_1,\ldots,\mathbf W_n) + (1-\alpha)\tilde\ell(\theta_2;\mathbf W_1,\ldots,\mathbf W_n)
\end{equation*}
that is the same as
\begin{equation*}
    \ell(\alpha h_1 + (1-\alpha)h_2;\mathbf W_1,\ldots,\mathbf W_n)
\geq
\alpha \ell(h_1;\mathbf W_1,\ldots,\mathbf W_n)
+ (1-\alpha)\ell(h_2;\mathbf W_1,\ldots,\mathbf W_n).
\end{equation*}

For the third property, we have
\begin{equation*}
    \Lambda(\alpha h_1 + (1-\alpha)h_2)
= \tilde\ell(\alpha\theta_1 + (1-\alpha)\theta_2).
\end{equation*}
By assumption (iii), the Hessian
$\nabla^2_{\theta}\tilde\Lambda(\theta)$
is strictly negative definite for all $\theta$.
Therefore, $\tilde\Lambda$ is strictly concave on $\mathbb R^{m+k+1}$, which implies that, for $\theta_1\neq\theta_2$,
\begin{equation*}
    \tilde\Lambda(\alpha\theta_1 + (1-\alpha)\theta_2)
    >
    \alpha \tilde\Lambda(\theta_1) + (1-\alpha)\tilde\Lambda(\theta_2)
\end{equation*}
that is the same as
\begin{equation*}
    \Lambda(\alpha h_1 + (1-\alpha)h_2)
>
\alpha \Lambda(h_1)
+ (1-\alpha)\Lambda(h_2),
\end{equation*}
for $h_1\neq h_2$.
\end{proof}