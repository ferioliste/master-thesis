% math sets
\newcommand{\N}{\mathbb N}
\newcommand{\Z}{\mathbb Z}
\newcommand{\Q}{\mathbb Q}
\newcommand{\R}{\mathbb R}
\newcommand{\C}{\mathbb C}

% define commands
\newcommand{\E}{\mathbb E}
\renewcommand{\Pr}{\mathbb P}
\newcommand{\var}{\mathbb{V}\mathrm{ar}}
\newcommand{\Cov}{\mathbb{C}\mathrm{ov}}
\newcommand{\I}{\mathbf{I}}
\newcommand{\tr}{\top}
\newcommand{\de}{\mathrm{d}}
\newcommand{\p}{\mathrm{p}}
\newcommand{\indep}{\perp\!\!\!\!\perp}
\newcommand{\bigo}{\mathcal{O}}

\DeclareMathOperator*{\argmin}{\mathrm{argmin}}
\DeclareMathOperator*{\argmax}{\mathrm{argmax}}

\usepackage{amsthm}

% Define a new theorem style: upright text + newline after heading
\newtheoremstyle{break}               % name
  {.5\baselineskip}                  % space above
  {.5\baselineskip}                  % space below
  {\normalfont}                       % body font (non-italic)
  {}                                  % indent amount
  {\bfseries}                         % theorem head font
  {}                                  % punctuation after theorem head
  {\newline}                          % space after theorem head (newline)
  {\thmname{#1}\ \thmnumber{#2}\thmnote{: #3}}                                  % theorem head spec

% Use the new style
\theoremstyle{break}
\newtheorem{theorem}{Theorem}[subsection]
\newtheorem{lemma}[theorem]{Lemma}
\newtheorem{proposition}[theorem]{Proposition}
\newtheorem{corollary}[theorem]{Corollary}
\newtheorem{definition}[theorem]{Definition}

\renewcommand*{\theoremautorefname}{Theorem}
\newcommand*{\lemmaautorefname}{Lemma}
\newcommand*{\propositionautorefname}{Proposition}
\newcommand*{\corollaryautorefname}{Corollary}
\newcommand*{\definitionautorefname}{Definition}